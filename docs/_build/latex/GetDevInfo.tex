%% Generated by Sphinx.
\def\sphinxdocclass{report}
\documentclass[letterpaper,10pt,english]{sphinxmanual}
\ifdefined\pdfpxdimen
   \let\sphinxpxdimen\pdfpxdimen\else\newdimen\sphinxpxdimen
\fi \sphinxpxdimen=.75bp\relax

\PassOptionsToPackage{warn}{textcomp}
\usepackage[utf8]{inputenc}
\ifdefined\DeclareUnicodeCharacter
% support both utf8 and utf8x syntaxes
\edef\sphinxdqmaybe{\ifdefined\DeclareUnicodeCharacterAsOptional\string"\fi}
  \DeclareUnicodeCharacter{\sphinxdqmaybe00A0}{\nobreakspace}
  \DeclareUnicodeCharacter{\sphinxdqmaybe2500}{\sphinxunichar{2500}}
  \DeclareUnicodeCharacter{\sphinxdqmaybe2502}{\sphinxunichar{2502}}
  \DeclareUnicodeCharacter{\sphinxdqmaybe2514}{\sphinxunichar{2514}}
  \DeclareUnicodeCharacter{\sphinxdqmaybe251C}{\sphinxunichar{251C}}
  \DeclareUnicodeCharacter{\sphinxdqmaybe2572}{\textbackslash}
\fi
\usepackage{cmap}
\usepackage[T1]{fontenc}
\usepackage{amsmath,amssymb,amstext}
\usepackage{babel}
\usepackage{times}
\usepackage[Bjarne]{fncychap}
\usepackage{sphinx}

\fvset{fontsize=\small}
\usepackage{geometry}

% Include hyperref last.
\usepackage{hyperref}
% Fix anchor placement for figures with captions.
\usepackage{hypcap}% it must be loaded after hyperref.
% Set up styles of URL: it should be placed after hyperref.
\urlstyle{same}

\addto\captionsenglish{\renewcommand{\figurename}{Fig.\@ }}
\makeatletter
\def\fnum@figure{\figurename\thefigure{}}
\makeatother
\addto\captionsenglish{\renewcommand{\tablename}{Table }}
\makeatletter
\def\fnum@table{\tablename\thetable{}}
\makeatother
\addto\captionsenglish{\renewcommand{\literalblockname}{Listing}}

\addto\captionsenglish{\renewcommand{\literalblockcontinuedname}{continued from previous page}}
\addto\captionsenglish{\renewcommand{\literalblockcontinuesname}{continues on next page}}
\addto\captionsenglish{\renewcommand{\sphinxnonalphabeticalgroupname}{Non-alphabetical}}
\addto\captionsenglish{\renewcommand{\sphinxsymbolsname}{Symbols}}
\addto\captionsenglish{\renewcommand{\sphinxnumbersname}{Numbers}}

\addto\extrasenglish{\def\pageautorefname{page}}

\setcounter{tocdepth}{1}



\title{GetDevInfo Documentation}
\date{Aug 10, 2020}
\release{1.1.0}
\author{Hamish McIntyre-Bhatty}
\newcommand{\sphinxlogo}{\vbox{}}
\renewcommand{\releasename}{Release}
\makeindex
\begin{document}

\pagestyle{empty}
\sphinxmaketitle
\pagestyle{plain}
\sphinxtableofcontents
\pagestyle{normal}
\phantomsection\label{\detokenize{index::doc}}


Contents:


\chapter{Documentation for the output format}
\label{\detokenize{format:documentation-for-the-output-format}}\label{\detokenize{format::doc}}
This module outputs data in a precisely-formatted dictionary object.
In order for it to be useful, this format, and the information that
is provided in it, needs to be explained precisely.

This format is the same on Linux, macOS, and Cygwin (Windows), but the
macOS and Cygwin versions of this library currently have less functionality,
so some of the information isn’t present on those platforms version. Instead,
placeholders like “N/A” or “Unknown” are used. Those instances will be pointed
out here.

\begin{sphinxadmonition}{note}{Note:}
On Linux and Cygwin, superuser/administrator privileges are required for
GetDevInfo to work correctly.
\end{sphinxadmonition}


\section{For each device and partition:}
\label{\detokenize{format:for-each-device-and-partition}}
A sub-dictionary is created with the name of that disk as its key.
\begin{description}
\item[{For example:}] \leavevmode
To access the info for /dev/disk1s1, use:

\begin{sphinxVerbatim}[commandchars=\\\{\}]
\PYG{g+gp}{\PYGZgt{}\PYGZgt{}\PYGZgt{} }\PYG{n}{DISKINFO}\PYG{p}{[}\PYG{l+s+s1}{\PYGZsq{}}\PYG{l+s+s1}{/dev/disk1s1}\PYG{l+s+s1}{\PYGZsq{}}\PYG{p}{]}
\end{sphinxVerbatim}

\end{description}


\section{Inside this sub-dictionary (standard devices):}
\label{\detokenize{format:inside-this-sub-dictionary-standard-devices}}
Various information is collected and organised here.
\begin{description}
\item[{‘Name’:}] \leavevmode
The disk’s name, stored as a string.

\item[{‘Type’:}] \leavevmode
Whether the disk is a “Device” or “Partition”, stored as a string.
\begin{description}
\item[{For example:}] \leavevmode
\begin{sphinxVerbatim}[commandchars=\\\{\}]
\PYG{g+gp}{\PYGZgt{}\PYGZgt{}\PYGZgt{} }\PYG{n}{DISKINFO}\PYG{p}{[}\PYG{l+s+s1}{\PYGZsq{}}\PYG{l+s+s1}{/dev/sda}\PYG{l+s+s1}{\PYGZsq{}}\PYG{p}{]}\PYG{p}{[}\PYG{l+s+s1}{\PYGZsq{}}\PYG{l+s+s1}{Type}\PYG{l+s+s1}{\PYGZsq{}}\PYG{p}{]}
\PYG{g+gp}{\PYGZgt{}\PYGZgt{}\PYGZgt{} }\PYG{l+s+s2}{\PYGZdq{}}\PYG{l+s+s2}{Device}\PYG{l+s+s2}{\PYGZdq{}}
\end{sphinxVerbatim}

\end{description}

\begin{sphinxadmonition}{note}{Note:}
Due to Cygwin limitations, all disks are considered devices on Cygwin.
\end{sphinxadmonition}

\item[{‘HostDevice’:}] \leavevmode
The “parent” or “host” device of a partition, stored as a string.
For a device, this is always set to “N/A”. For an LVM disk, this is
the host device of the containing partition. eg: /dev/sdb.
\begin{description}
\item[{Example 1:}] \leavevmode
\begin{sphinxVerbatim}[commandchars=\\\{\}]
\PYG{g+gp}{\PYGZgt{}\PYGZgt{}\PYGZgt{} }\PYG{n}{DISKINFO}\PYG{p}{[}\PYG{l+s+s1}{\PYGZsq{}}\PYG{l+s+s1}{/dev/sda}\PYG{l+s+s1}{\PYGZsq{}}\PYG{p}{]}\PYG{p}{[}\PYG{l+s+s1}{\PYGZsq{}}\PYG{l+s+s1}{HostDevice}\PYG{l+s+s1}{\PYGZsq{}}\PYG{p}{]}
\PYG{g+gp}{\PYGZgt{}\PYGZgt{}\PYGZgt{} }\PYG{l+s+s2}{\PYGZdq{}}\PYG{l+s+s2}{N/A}\PYG{l+s+s2}{\PYGZdq{}}
\end{sphinxVerbatim}

\item[{Example 2:}] \leavevmode
\begin{sphinxVerbatim}[commandchars=\\\{\}]
\PYG{g+gp}{\PYGZgt{}\PYGZgt{}\PYGZgt{} }\PYG{n}{DISKINFO}\PYG{p}{[}\PYG{l+s+s1}{\PYGZsq{}}\PYG{l+s+s1}{/dev/sde5}\PYG{l+s+s1}{\PYGZsq{}}\PYG{p}{]}\PYG{p}{[}\PYG{l+s+s1}{\PYGZsq{}}\PYG{l+s+s1}{HostDevice}\PYG{l+s+s1}{\PYGZsq{}}\PYG{p}{]}
\PYG{g+gp}{\PYGZgt{}\PYGZgt{}\PYGZgt{} }\PYG{l+s+s2}{\PYGZdq{}}\PYG{l+s+s2}{/dev/sde}\PYG{l+s+s2}{\PYGZdq{}}
\end{sphinxVerbatim}

\item[{Example 3:}] \leavevmode
\begin{sphinxVerbatim}[commandchars=\\\{\}]
\PYG{g+gp}{\PYGZgt{}\PYGZgt{}\PYGZgt{} }\PYG{n}{DISKINFO}\PYG{p}{[}\PYG{l+s+s1}{\PYGZsq{}}\PYG{l+s+s1}{/dev/disk1s3}\PYG{l+s+s1}{\PYGZsq{}}\PYG{p}{]}\PYG{p}{[}\PYG{l+s+s1}{\PYGZsq{}}\PYG{l+s+s1}{HostDevice}\PYG{l+s+s1}{\PYGZsq{}}\PYG{p}{]}
\PYG{g+gp}{\PYGZgt{}\PYGZgt{}\PYGZgt{} }\PYG{l+s+s2}{\PYGZdq{}}\PYG{l+s+s2}{/dev/disk1}\PYG{l+s+s2}{\PYGZdq{}}
\end{sphinxVerbatim}

\end{description}

\item[{‘Partitions’:}] \leavevmode
All the partitions a device contains, stored as a list. For partitions,
this is always set to {[}{]}.
\begin{description}
\item[{Example 1:}] \leavevmode
\begin{sphinxVerbatim}[commandchars=\\\{\}]
\PYG{g+gp}{\PYGZgt{}\PYGZgt{}\PYGZgt{} }\PYG{n}{DISKINFO}\PYG{p}{[}\PYG{l+s+s1}{\PYGZsq{}}\PYG{l+s+s1}{/dev/sda1}\PYG{l+s+s1}{\PYGZsq{}}\PYG{p}{]}\PYG{p}{[}\PYG{l+s+s1}{\PYGZsq{}}\PYG{l+s+s1}{Partitions}\PYG{l+s+s1}{\PYGZsq{}}\PYG{p}{]}
\PYG{g+gp}{\PYGZgt{}\PYGZgt{}\PYGZgt{} }\PYG{p}{[}\PYG{p}{]}
\end{sphinxVerbatim}

\item[{Example 2:}] \leavevmode
\begin{sphinxVerbatim}[commandchars=\\\{\}]
\PYG{g+gp}{\PYGZgt{}\PYGZgt{}\PYGZgt{} }\PYG{n}{DISKINFO}\PYG{p}{[}\PYG{l+s+s1}{\PYGZsq{}}\PYG{l+s+s1}{/dev/sda}\PYG{l+s+s1}{\PYGZsq{}}\PYG{p}{]}\PYG{p}{[}\PYG{l+s+s1}{\PYGZsq{}}\PYG{l+s+s1}{Partitions}\PYG{l+s+s1}{\PYGZsq{}}\PYG{p}{]}
\PYG{g+gp}{\PYGZgt{}\PYGZgt{}\PYGZgt{} }\PYG{p}{[}\PYG{l+s+s2}{\PYGZdq{}}\PYG{l+s+s2}{/dev/sda1}\PYG{l+s+s2}{\PYGZdq{}}\PYG{p}{,} \PYG{l+s+s2}{\PYGZdq{}}\PYG{l+s+s2}{/dev/sda2}\PYG{l+s+s2}{\PYGZdq{}}\PYG{p}{,} \PYG{l+s+s2}{\PYGZdq{}}\PYG{l+s+s2}{/dev/sda3}\PYG{l+s+s2}{\PYGZdq{}}\PYG{p}{]}
\end{sphinxVerbatim}

\item[{Example 3:}] \leavevmode
\begin{sphinxVerbatim}[commandchars=\\\{\}]
\PYG{g+gp}{\PYGZgt{}\PYGZgt{}\PYGZgt{} }\PYG{n}{DISKINFO}\PYG{p}{[}\PYG{l+s+s1}{\PYGZsq{}}\PYG{l+s+s1}{/dev/disk0}\PYG{l+s+s1}{\PYGZsq{}}\PYG{p}{]}\PYG{p}{[}\PYG{l+s+s1}{\PYGZsq{}}\PYG{l+s+s1}{Partitions}\PYG{l+s+s1}{\PYGZsq{}}\PYG{p}{]}
\PYG{g+gp}{\PYGZgt{}\PYGZgt{}\PYGZgt{} }\PYG{p}{[}\PYG{l+s+s2}{\PYGZdq{}}\PYG{l+s+s2}{/dev/disk0s1}\PYG{l+s+s2}{\PYGZdq{}}\PYG{p}{,} \PYG{l+s+s2}{\PYGZdq{}}\PYG{l+s+s2}{/dev/disk0s2}\PYG{l+s+s2}{\PYGZdq{}}\PYG{p}{]}
\end{sphinxVerbatim}

\end{description}

\begin{sphinxadmonition}{note}{Note:}
Not yet available on Cygwin.
\end{sphinxadmonition}

\item[{‘Vendor’:}] \leavevmode
The device’s/partition’s vendor. For a device, this is often the brand. For
partitions this is more random, but often has something to do with the
file system type, or the OS that created the partition.
\begin{description}
\item[{Example 1:}] \leavevmode
\begin{sphinxVerbatim}[commandchars=\\\{\}]
\PYG{g+gp}{\PYGZgt{}\PYGZgt{}\PYGZgt{} }\PYG{n}{DISKINFO}\PYG{p}{[}\PYG{l+s+s1}{\PYGZsq{}}\PYG{l+s+s1}{/dev/sda}\PYG{l+s+s1}{\PYGZsq{}}\PYG{p}{]}\PYG{p}{[}\PYG{l+s+s1}{\PYGZsq{}}\PYG{l+s+s1}{Vendor}\PYG{l+s+s1}{\PYGZsq{}}\PYG{p}{]}
\PYG{g+gp}{\PYGZgt{}\PYGZgt{}\PYGZgt{} }\PYG{l+s+s2}{\PYGZdq{}}\PYG{l+s+s2}{VBOX}\PYG{l+s+s2}{\PYGZdq{}}
\end{sphinxVerbatim}

\item[{Example 2:}] \leavevmode
\begin{sphinxVerbatim}[commandchars=\\\{\}]
\PYG{g+gp}{\PYGZgt{}\PYGZgt{}\PYGZgt{} }\PYG{n}{DISKINFO}\PYG{p}{[}\PYG{l+s+s1}{\PYGZsq{}}\PYG{l+s+s1}{/dev/sda1}\PYG{l+s+s1}{\PYGZsq{}}\PYG{p}{]}\PYG{p}{[}\PYG{l+s+s1}{\PYGZsq{}}\PYG{l+s+s1}{Vendor}\PYG{l+s+s1}{\PYGZsq{}}\PYG{p}{]}
\PYG{g+gp}{\PYGZgt{}\PYGZgt{}\PYGZgt{} }\PYG{l+s+s2}{\PYGZdq{}}\PYG{l+s+s2}{Linux}\PYG{l+s+s2}{\PYGZdq{}}
\end{sphinxVerbatim}

\item[{Example 3:}] \leavevmode
\begin{sphinxVerbatim}[commandchars=\\\{\}]
\PYG{g+gp}{\PYGZgt{}\PYGZgt{}\PYGZgt{} }\PYG{n}{DISKINFO}\PYG{p}{[}\PYG{l+s+s1}{\PYGZsq{}}\PYG{l+s+s1}{/dev/disk0s1}\PYG{l+s+s1}{\PYGZsq{}}\PYG{p}{]}\PYG{p}{[}\PYG{l+s+s1}{\PYGZsq{}}\PYG{l+s+s1}{Vendor}\PYG{l+s+s1}{\PYGZsq{}}\PYG{p}{]}
\PYG{g+gp}{\PYGZgt{}\PYGZgt{}\PYGZgt{} }\PYG{l+s+s2}{\PYGZdq{}}\PYG{l+s+s2}{VBOX}\PYG{l+s+s2}{\PYGZdq{}}
\end{sphinxVerbatim}

\end{description}

\begin{sphinxadmonition}{note}{Note:}
Not available for all disks yet on Cygwin.
\end{sphinxadmonition}

\item[{‘Product’:}] \leavevmode
The device’s product information. Often model information such as a model
name/number. For a partition, this is always the same as it’s host device’s
product information, prefixed by “Host Device: “.
\begin{description}
\item[{Example 1:}] \leavevmode
\begin{sphinxVerbatim}[commandchars=\\\{\}]
\PYG{g+gp}{\PYGZgt{}\PYGZgt{}\PYGZgt{} }\PYG{n}{DISKINFO}\PYG{p}{[}\PYG{l+s+s1}{\PYGZsq{}}\PYG{l+s+s1}{/dev/sda}\PYG{l+s+s1}{\PYGZsq{}}\PYG{p}{]}\PYG{p}{[}\PYG{l+s+s1}{\PYGZsq{}}\PYG{l+s+s1}{Product}\PYG{l+s+s1}{\PYGZsq{}}\PYG{p}{]}
\PYG{g+gp}{\PYGZgt{}\PYGZgt{}\PYGZgt{} }\PYG{l+s+s2}{\PYGZdq{}}\PYG{l+s+s2}{ST1000DM003\PYGZhy{}1CH1}\PYG{l+s+s2}{\PYGZdq{}}
\end{sphinxVerbatim}

\item[{Example 2:}] \leavevmode
\begin{sphinxVerbatim}[commandchars=\\\{\}]
\PYG{g+gp}{\PYGZgt{}\PYGZgt{}\PYGZgt{} }\PYG{n}{DISKINFO}\PYG{p}{[}\PYG{l+s+s1}{\PYGZsq{}}\PYG{l+s+s1}{/dev/sda1}\PYG{l+s+s1}{\PYGZsq{}}\PYG{p}{]}\PYG{p}{[}\PYG{l+s+s1}{\PYGZsq{}}\PYG{l+s+s1}{Product}\PYG{l+s+s1}{\PYGZsq{}}\PYG{p}{]}
\PYG{g+gp}{\PYGZgt{}\PYGZgt{}\PYGZgt{} }\PYG{l+s+s2}{\PYGZdq{}}\PYG{l+s+s2}{Host Device: ST1000DM003\PYGZhy{}1CH1}\PYG{l+s+s2}{\PYGZdq{}}
\end{sphinxVerbatim}

\item[{Example 3:}] \leavevmode
\begin{sphinxVerbatim}[commandchars=\\\{\}]
\PYG{g+gp}{\PYGZgt{}\PYGZgt{}\PYGZgt{} }\PYG{n}{DISKINFO}\PYG{p}{[}\PYG{l+s+s1}{\PYGZsq{}}\PYG{l+s+s1}{/dev/disk0}\PYG{l+s+s1}{\PYGZsq{}}\PYG{p}{]}\PYG{p}{[}\PYG{l+s+s1}{\PYGZsq{}}\PYG{l+s+s1}{Product}\PYG{l+s+s1}{\PYGZsq{}}\PYG{p}{]}
\PYG{g+gp}{\PYGZgt{}\PYGZgt{}\PYGZgt{} }\PYG{l+s+s2}{\PYGZdq{}}\PYG{l+s+s2}{HARDDISK}\PYG{l+s+s2}{\PYGZdq{}}
\end{sphinxVerbatim}

\end{description}

\begin{sphinxadmonition}{note}{Note:}
Not available for all disks yet on Cygwin.
\end{sphinxadmonition}

\item[{‘Capacity’, and ‘RawCapacity’:}] \leavevmode
The disk’s capacity, in both human-readable form, and program-friendly form.
Ignored for some types of disks, like optical drives. The human-readable
capacity is rounded to make it a 3 digit number. The machine-readable size is
measured in bytes, and it is not rounded.

\begin{sphinxadmonition}{note}{Note:}
Not available for all disks yet on Cygwin.
\end{sphinxadmonition}
\begin{description}
\item[{Example:}] \leavevmode
\begin{sphinxVerbatim}[commandchars=\\\{\}]
\PYG{g+gp}{\PYGZgt{}\PYGZgt{}\PYGZgt{} }\PYG{n}{DISKINFO}\PYG{p}{[}\PYG{l+s+s1}{\PYGZsq{}}\PYG{l+s+s1}{/dev/sda}\PYG{l+s+s1}{\PYGZsq{}}\PYG{p}{]}\PYG{p}{[}\PYG{l+s+s1}{\PYGZsq{}}\PYG{l+s+s1}{Capacity}\PYG{l+s+s1}{\PYGZsq{}}\PYG{p}{]}
\PYG{g+gp}{\PYGZgt{}\PYGZgt{}\PYGZgt{} }\PYG{l+s+s2}{\PYGZdq{}}\PYG{l+s+s2}{500 GB}\PYG{l+s+s2}{\PYGZdq{}}
\end{sphinxVerbatim}

\begin{sphinxVerbatim}[commandchars=\\\{\}]
\PYG{g+gp}{\PYGZgt{}\PYGZgt{}\PYGZgt{} }\PYG{n}{DISKINFO}\PYG{p}{[}\PYG{l+s+s1}{\PYGZsq{}}\PYG{l+s+s1}{/dev/sda}\PYG{l+s+s1}{\PYGZsq{}}\PYG{p}{]}\PYG{p}{[}\PYG{l+s+s1}{\PYGZsq{}}\PYG{l+s+s1}{RawCapacity}\PYG{l+s+s1}{\PYGZsq{}}\PYG{p}{]}
\PYG{g+gp}{\PYGZgt{}\PYGZgt{}\PYGZgt{} }\PYG{l+s+s2}{\PYGZdq{}}\PYG{l+s+s2}{500107862016}\PYG{l+s+s2}{\PYGZdq{}}
\end{sphinxVerbatim}

\end{description}

\item[{‘Description’:}] \leavevmode
A human-readable description of the disk. Simply here to make it easier
for a human to identify a disk. On Linux, these are the descriptions provided by
lshw (except for logical volumes), and they are fairly basic. On macOS, these are
generated using information from diskutil. On Cygwin, these are generated and provide
information like the drive letter and bus used (eg ATA).
\begin{description}
\item[{Example 1:}] \leavevmode
\begin{sphinxVerbatim}[commandchars=\\\{\}]
\PYG{g+gp}{\PYGZgt{}\PYGZgt{}\PYGZgt{} }\PYG{n}{DISKINFO}\PYG{p}{[}\PYG{l+s+s1}{\PYGZsq{}}\PYG{l+s+s1}{/dev/sda}\PYG{l+s+s1}{\PYGZsq{}}\PYG{p}{]}\PYG{p}{[}\PYG{l+s+s1}{\PYGZsq{}}\PYG{l+s+s1}{Description}\PYG{l+s+s1}{\PYGZsq{}}\PYG{p}{]}
\PYG{g+gp}{\PYGZgt{}\PYGZgt{}\PYGZgt{} }\PYG{l+s+s2}{\PYGZdq{}}\PYG{l+s+s2}{ATA Disk}\PYG{l+s+s2}{\PYGZdq{}}
\end{sphinxVerbatim}

\item[{Example 2:}] \leavevmode
\begin{sphinxVerbatim}[commandchars=\\\{\}]
\PYG{g+gp}{\PYGZgt{}\PYGZgt{}\PYGZgt{} }\PYG{n}{DISKINFO}\PYG{p}{[}\PYG{l+s+s1}{\PYGZsq{}}\PYG{l+s+s1}{/dev/disk1}\PYG{l+s+s1}{\PYGZsq{}}\PYG{p}{]}\PYG{p}{[}\PYG{l+s+s1}{\PYGZsq{}}\PYG{l+s+s1}{Description}\PYG{l+s+s1}{\PYGZsq{}}\PYG{p}{]}
\PYG{g+gp}{\PYGZgt{}\PYGZgt{}\PYGZgt{} }\PYG{l+s+s2}{\PYGZdq{}}\PYG{l+s+s2}{Internal Hard Disk Drive (Connected through SATA)}\PYG{l+s+s2}{\PYGZdq{}}
\end{sphinxVerbatim}

\end{description}

\item[{‘Flags’:}] \leavevmode
The disk’s capabilities, stored as a list.

\begin{sphinxadmonition}{note}{Note:}
Not yet available on macOS, Cygwin, or for logical volumes on Linux.
\end{sphinxadmonition}
\begin{description}
\item[{For example:}] \leavevmode
\begin{sphinxVerbatim}[commandchars=\\\{\}]
\PYG{g+gp}{\PYGZgt{}\PYGZgt{}\PYGZgt{} }\PYG{n}{DISKINFO}\PYG{p}{[}\PYG{l+s+s1}{\PYGZsq{}}\PYG{l+s+s1}{/dev/cdrom}\PYG{l+s+s1}{\PYGZsq{}}\PYG{p}{]}\PYG{p}{[}\PYG{l+s+s1}{\PYGZsq{}}\PYG{l+s+s1}{Flags}\PYG{l+s+s1}{\PYGZsq{}}\PYG{p}{]}
\PYG{g+gp}{\PYGZgt{}\PYGZgt{}\PYGZgt{} }\PYG{p}{[}\PYG{l+s+s1}{\PYGZsq{}}\PYG{l+s+s1}{removable}\PYG{l+s+s1}{\PYGZsq{}}\PYG{p}{,} \PYG{l+s+s1}{\PYGZsq{}}\PYG{l+s+s1}{audio}\PYG{l+s+s1}{\PYGZsq{}}\PYG{p}{,} \PYG{l+s+s1}{\PYGZsq{}}\PYG{l+s+s1}{cd\PYGZhy{}r}\PYG{l+s+s1}{\PYGZsq{}}\PYG{p}{,} \PYG{l+s+s1}{\PYGZsq{}}\PYG{l+s+s1}{cd\PYGZhy{}rw}\PYG{l+s+s1}{\PYGZsq{}}\PYG{p}{,} \PYG{l+s+s1}{\PYGZsq{}}\PYG{l+s+s1}{dvd}\PYG{l+s+s1}{\PYGZsq{}}\PYG{p}{,} \PYG{l+s+s1}{\PYGZsq{}}\PYG{l+s+s1}{dvd\PYGZhy{}r}\PYG{l+s+s1}{\PYGZsq{}}\PYG{p}{,} \PYG{l+s+s1}{\PYGZsq{}}\PYG{l+s+s1}{dvd\PYGZhy{}ram}\PYG{l+s+s1}{\PYGZsq{}}\PYG{p}{]}
\end{sphinxVerbatim}

\end{description}

\item[{‘Partitioning’:}] \leavevmode
The disk’s partition scheme. N/A for partitions and logical volumes.

\begin{sphinxadmonition}{note}{Note:}
Not yet available on macOS.
\end{sphinxadmonition}
\begin{description}
\item[{Example 1:}] \leavevmode
\begin{sphinxVerbatim}[commandchars=\\\{\}]
\PYG{g+gp}{\PYGZgt{}\PYGZgt{}\PYGZgt{} }\PYG{n}{DISKINFO}\PYG{p}{[}\PYG{l+s+s1}{\PYGZsq{}}\PYG{l+s+s1}{/dev/sda}\PYG{l+s+s1}{\PYGZsq{}}\PYG{p}{]}\PYG{p}{[}\PYG{l+s+s1}{\PYGZsq{}}\PYG{l+s+s1}{Partitioning}\PYG{l+s+s1}{\PYGZsq{}}\PYG{p}{]}
\PYG{g+gp}{\PYGZgt{}\PYGZgt{}\PYGZgt{} }\PYG{l+s+s2}{\PYGZdq{}}\PYG{l+s+s2}{gpt}\PYG{l+s+s2}{\PYGZdq{}}
\end{sphinxVerbatim}

\item[{Example 2:}] \leavevmode
\begin{sphinxVerbatim}[commandchars=\\\{\}]
\PYG{g+gp}{\PYGZgt{}\PYGZgt{}\PYGZgt{} }\PYG{n}{DISKINFO}\PYG{p}{[}\PYG{l+s+s1}{\PYGZsq{}}\PYG{l+s+s1}{/dev/sdb}\PYG{l+s+s1}{\PYGZsq{}}\PYG{p}{]}\PYG{p}{[}\PYG{l+s+s1}{\PYGZsq{}}\PYG{l+s+s1}{Partitioning}\PYG{l+s+s1}{\PYGZsq{}}\PYG{p}{]}
\PYG{g+gp}{\PYGZgt{}\PYGZgt{}\PYGZgt{} }\PYG{l+s+s2}{\PYGZdq{}}\PYG{l+s+s2}{mbr}\PYG{l+s+s2}{\PYGZdq{}}
\end{sphinxVerbatim}

\end{description}

\item[{‘FileSystem’:}] \leavevmode
The disk’s file system. N/A for devices.

\begin{sphinxadmonition}{note}{Note:}
Not yet available on macOS.
\end{sphinxadmonition}
\begin{description}
\item[{Example:}] \leavevmode
\begin{sphinxVerbatim}[commandchars=\\\{\}]
\PYG{g+gp}{\PYGZgt{}\PYGZgt{}\PYGZgt{} }\PYG{n}{DISKINFO}\PYG{p}{[}\PYG{l+s+s1}{\PYGZsq{}}\PYG{l+s+s1}{/dev/sda}\PYG{l+s+s1}{\PYGZsq{}}\PYG{p}{]}\PYG{p}{[}\PYG{l+s+s1}{\PYGZsq{}}\PYG{l+s+s1}{FileSystem}\PYG{l+s+s1}{\PYGZsq{}}\PYG{p}{]}
\PYG{g+gp}{\PYGZgt{}\PYGZgt{}\PYGZgt{} }\PYG{l+s+s2}{\PYGZdq{}}\PYG{l+s+s2}{ext4}\PYG{l+s+s2}{\PYGZdq{}}
\end{sphinxVerbatim}

\end{description}

\item[{‘UUID’:}] \leavevmode
This disk’s UUID. N/A for devices. Length changes based on filesystem
type. For example, vfat UUIDs are shorter.

\begin{sphinxadmonition}{note}{Note:}
Not yet available on macOS.
\end{sphinxadmonition}
\begin{description}
\item[{Example:}] \leavevmode
\begin{sphinxVerbatim}[commandchars=\\\{\}]
\PYG{g+gp}{\PYGZgt{}\PYGZgt{}\PYGZgt{} }\PYG{n}{DISKINFO}\PYG{p}{[}\PYG{l+s+s1}{\PYGZsq{}}\PYG{l+s+s1}{/dev/sda1}\PYG{l+s+s1}{\PYGZsq{}}\PYG{p}{]}\PYG{p}{[}\PYG{l+s+s1}{\PYGZsq{}}\PYG{l+s+s1}{UUID}\PYG{l+s+s1}{\PYGZsq{}}\PYG{p}{]}
\PYG{g+gp}{\PYGZgt{}\PYGZgt{}\PYGZgt{} }\PYG{n}{XXXX}\PYG{o}{\PYGZhy{}}\PYG{n}{XXXX}
\end{sphinxVerbatim}

\end{description}

\item[{‘ID’:}] \leavevmode
The disk’s ID.

\begin{sphinxadmonition}{note}{Note:}
Not yet available on macOS or Cygwin.
\end{sphinxadmonition}
\begin{description}
\item[{Example:}] \leavevmode
\begin{sphinxVerbatim}[commandchars=\\\{\}]
\PYG{g+gp}{\PYGZgt{}\PYGZgt{}\PYGZgt{} }\PYG{n}{DISKINFO}\PYG{p}{[}\PYG{l+s+s1}{\PYGZsq{}}\PYG{l+s+s1}{/dev/sda}\PYG{l+s+s1}{\PYGZsq{}}\PYG{p}{]}\PYG{p}{[}\PYG{l+s+s1}{\PYGZsq{}}\PYG{l+s+s1}{ID}\PYG{l+s+s1}{\PYGZsq{}}\PYG{p}{]}
\PYG{g+gp}{\PYGZgt{}\PYGZgt{}\PYGZgt{} }\PYG{l+s+s2}{\PYGZdq{}}\PYG{l+s+s2}{usb\PYGZhy{}Generic\PYGZus{}STORAGE\PYGZus{}DEVICE\PYGZus{}000000001206\PYGZhy{}0:1}\PYG{l+s+s2}{\PYGZdq{}}
\end{sphinxVerbatim}

\end{description}

\item[{‘BootRecord’, ‘BootRecordStrings’:}] \leavevmode
The MBR/PBR of the disk. Can be useful in identifying the bootloader that
resides there, if any.

\begin{sphinxadmonition}{note}{Note:}
Not yet available on macOS.
\end{sphinxadmonition}

\end{description}


\section{Inside this sub-dictionary (specifics for LVM disks on Linux):}
\label{\detokenize{format:inside-this-sub-dictionary-specifics-for-lvm-disks-on-linux}}
These are keys that are only present for LVM disks (where “Product” is “LVM Partition”).
\begin{description}
\item[{‘Aliases’:}] \leavevmode
Any aliases the disk has. LVM disks can often be accessed using multiple
different names. This is a list of those names.
\begin{description}
\item[{Example:}] \leavevmode
\begin{sphinxVerbatim}[commandchars=\\\{\}]
\PYG{g+gp}{\PYGZgt{}\PYGZgt{}\PYGZgt{} }\PYG{n}{DISKINFO}\PYG{p}{[}\PYG{l+s+s1}{\PYGZsq{}}\PYG{l+s+s1}{/dev/mapper/fedora/root}\PYG{l+s+s1}{\PYGZsq{}}\PYG{p}{]}\PYG{p}{[}\PYG{l+s+s1}{\PYGZsq{}}\PYG{l+s+s1}{Aliases}\PYG{l+s+s1}{\PYGZsq{}}\PYG{p}{]}
\PYG{g+gp}{\PYGZgt{}\PYGZgt{}\PYGZgt{} }\PYG{p}{[}\PYG{l+s+s1}{\PYGZsq{}}\PYG{l+s+s1}{/dev/mapper/fedora/root}\PYG{l+s+s1}{\PYGZsq{}}\PYG{p}{,} \PYG{l+s+s1}{\PYGZsq{}}\PYG{l+s+s1}{/dev/fedora\PYGZhy{}\PYGZhy{}localhost\PYGZhy{}root}\PYG{l+s+s1}{\PYGZsq{}}\PYG{p}{]}
\end{sphinxVerbatim}

\end{description}

\item[{‘LVName’:}] \leavevmode
The name of the logical volume.
\begin{description}
\item[{Example:}] \leavevmode
\begin{sphinxVerbatim}[commandchars=\\\{\}]
\PYG{g+gp}{\PYGZgt{}\PYGZgt{}\PYGZgt{} }\PYG{n}{DISKINFO}\PYG{p}{[}\PYG{l+s+s1}{\PYGZsq{}}\PYG{l+s+s1}{/dev/mapper/fedora/root}\PYG{l+s+s1}{\PYGZsq{}}\PYG{p}{]}\PYG{p}{[}\PYG{l+s+s1}{\PYGZsq{}}\PYG{l+s+s1}{LVName}\PYG{l+s+s1}{\PYGZsq{}}\PYG{p}{]}
\PYG{g+gp}{\PYGZgt{}\PYGZgt{}\PYGZgt{} }\PYG{l+s+s2}{\PYGZdq{}}\PYG{l+s+s2}{root}\PYG{l+s+s2}{\PYGZdq{}}
\end{sphinxVerbatim}

\end{description}

\item[{‘VGName’:}] \leavevmode
The name of the volume group the logical volume belongs to.
\begin{description}
\item[{Example:}] \leavevmode
\begin{sphinxVerbatim}[commandchars=\\\{\}]
\PYG{g+gp}{\PYGZgt{}\PYGZgt{}\PYGZgt{} }\PYG{n}{DISKINFO}\PYG{p}{[}\PYG{l+s+s1}{\PYGZsq{}}\PYG{l+s+s1}{/dev/mapper/fedora/root}\PYG{l+s+s1}{\PYGZsq{}}\PYG{p}{]}\PYG{p}{[}\PYG{l+s+s1}{\PYGZsq{}}\PYG{l+s+s1}{VGName}\PYG{l+s+s1}{\PYGZsq{}}\PYG{p}{]}
\PYG{g+gp}{\PYGZgt{}\PYGZgt{}\PYGZgt{} }\PYG{l+s+s2}{\PYGZdq{}}\PYG{l+s+s2}{fedora}\PYG{l+s+s2}{\PYGZdq{}}
\end{sphinxVerbatim}

\end{description}

\item[{‘HostPartition’:}] \leavevmode
The partition that contains this logical volume.
\begin{description}
\item[{Example:}] \leavevmode
\begin{sphinxVerbatim}[commandchars=\\\{\}]
\PYG{g+gp}{\PYGZgt{}\PYGZgt{}\PYGZgt{} }\PYG{n}{DISKINFO}\PYG{p}{[}\PYG{l+s+s1}{\PYGZsq{}}\PYG{l+s+s1}{/dev/mapper/fedora/root}\PYG{l+s+s1}{\PYGZsq{}}\PYG{p}{]}\PYG{p}{[}\PYG{l+s+s1}{\PYGZsq{}}\PYG{l+s+s1}{HostPartition}\PYG{l+s+s1}{\PYGZsq{}}\PYG{p}{]}
\PYG{g+gp}{\PYGZgt{}\PYGZgt{}\PYGZgt{} }\PYG{l+s+s2}{\PYGZdq{}}\PYG{l+s+s2}{/dev/sda}\PYG{l+s+s2}{\PYGZdq{}}
\end{sphinxVerbatim}

\end{description}

\begin{sphinxadmonition}{note}{Note:}
Not always available depending on disk configuration.
\end{sphinxadmonition}

\end{description}

\begin{sphinxadmonition}{warning}{Warning:}
“UUID” may or may not be available for certain disks.
\end{sphinxadmonition}

\begin{sphinxadmonition}{warning}{Warning:}
“Capacity” and “RawCapacity” may not be available for certain disks.
\end{sphinxadmonition}

\begin{sphinxadmonition}{warning}{Warning:}
“HostPartition” and “HostDevice” may not be available for certain disks.
\end{sphinxadmonition}


\section{Inside this sub-dictionary (NVME disks):}
\label{\detokenize{format:inside-this-sub-dictionary-nvme-disks}}
\begin{sphinxadmonition}{warning}{Warning:}
Various standard keys are not available for NVME disks as they aren’t supported by lshw.
\end{sphinxadmonition}


\chapter{Documentation for the getdevinfo module}
\label{\detokenize{getdevinfo:module-getdevinfo.getdevinfo}}\label{\detokenize{getdevinfo:documentation-for-the-getdevinfo-module}}\label{\detokenize{getdevinfo::doc}}\index{getdevinfo.getdevinfo (module)@\spxentry{getdevinfo.getdevinfo}\spxextra{module}}
This is the part of the package that you would normally import and use.
It detects your platform (Linux or macOS), and runs the correct tools
for that platform.

For example:

\begin{sphinxVerbatim}[commandchars=\\\{\}]
\PYG{g+gp}{\PYGZgt{}\PYGZgt{}\PYGZgt{} }\PYG{k+kn}{import} \PYG{n+nn}{getdevinfo}
\PYG{g+gp}{\PYGZgt{}\PYGZgt{}\PYGZgt{} }\PYG{n}{getdevinfo}\PYG{o}{.}\PYG{n}{getdevinfo}\PYG{o}{.}\PYG{n}{get\PYGZus{}info}\PYG{p}{(}\PYG{p}{)}
\end{sphinxVerbatim}

Or, more concisely:

\begin{sphinxVerbatim}[commandchars=\\\{\}]
\PYG{g+gp}{\PYGZgt{}\PYGZgt{}\PYGZgt{} }\PYG{k+kn}{import} \PYG{n+nn}{getdevinfo}\PYG{n+nn}{.}\PYG{n+nn}{getdevinfo} \PYG{k}{as} \PYG{n+nn}{getdevinfo}
\PYG{g+gp}{\PYGZgt{}\PYGZgt{}\PYGZgt{} }\PYG{n}{getdevinfo}\PYG{o}{.}\PYG{n}{get\PYGZus{}info}\PYG{p}{(}\PYG{p}{)}
\end{sphinxVerbatim}

Will run the correct tools for your platform and return the collected
disk information as a dictionary.

\begin{sphinxadmonition}{note}{Note:}
You can import the submodules directly, but this might result
in strange behaviour, or not work on your platform if you
import the wrong one.  That is not how the package is intended
to be used, except if you want to use the get\_block\_size()
function to get a block size, as documented for each platform
later.
\end{sphinxadmonition}
\index{get\_info() (in module getdevinfo.getdevinfo)@\spxentry{get\_info()}\spxextra{in module getdevinfo.getdevinfo}}

\begin{fulllineitems}
\phantomsection\label{\detokenize{getdevinfo:getdevinfo.getdevinfo.get_info}}\pysiglinewithargsret{\sphinxcode{\sphinxupquote{getdevinfo.getdevinfo.}}\sphinxbfcode{\sphinxupquote{get\_info}}}{}{}
This function is used to determine the platform you’re using
(Linux or macOS) and run the relevant tools. Then, it returns
the disk information dictionary to the caller.
\begin{description}
\item[{Returns:}] \leavevmode
dict, the disk info dictionary.

\item[{Raises:}] \leavevmode
Hopefully nothing, but if there is an unhandled error or
bug elsewhere, there’s a small chance it could propagate
to here. If this concerns you, you can wrap this code in
a try:, except: clause:

\begin{sphinxVerbatim}[commandchars=\\\{\}]
\PYG{g+gp}{\PYGZgt{}\PYGZgt{}\PYGZgt{} }\PYG{k}{try}\PYG{p}{:}
\PYG{g+gp}{\PYGZgt{}\PYGZgt{}\PYGZgt{} }    \PYG{n}{get\PYGZus{}info}\PYG{p}{(}\PYG{p}{)}
\PYG{g+gp}{\PYGZgt{}\PYGZgt{}\PYGZgt{} }\PYG{k}{except}\PYG{p}{:}
\PYG{g+gp}{\PYGZgt{}\PYGZgt{}\PYGZgt{} }    \PYG{c+c1}{\PYGZsh{}Handle the error.}
\end{sphinxVerbatim}

\end{description}

Usage:

\begin{sphinxVerbatim}[commandchars=\\\{\}]
\PYG{g+gp}{\PYGZgt{}\PYGZgt{}\PYGZgt{} }\PYG{n}{disk\PYGZus{}info} \PYG{o}{=} \PYG{n}{get\PYGZus{}info}\PYG{p}{(}\PYG{p}{)}
\end{sphinxVerbatim}

\end{fulllineitems}



\chapter{Documentation for the linux module}
\label{\detokenize{linux:module-getdevinfo.linux}}\label{\detokenize{linux:documentation-for-the-linux-module}}\label{\detokenize{linux::doc}}\index{getdevinfo.linux (module)@\spxentry{getdevinfo.linux}\spxextra{module}}
This is the part of the package that contains the tools and information
getters for Linux. This would normally be called from the getdevinfo
module, but you can call it directly if you like.

\begin{sphinxadmonition}{note}{Note:}
You can import this submodule directly, but it might result
in strange behaviour, or not work on your platform if you
import the wrong one. That is not how the package is intended
to be used, except if you want to use the get\_block\_size()
function to get a block size, as documented below.
\end{sphinxadmonition}

\begin{sphinxadmonition}{warning}{Warning:}
Feel free to experiment, but be aware that you may be able to
cause crashes, exceptions, and generally weird situations by calling
these methods directly if you get it wrong. A good place to
look if you’re interested in this is the unit tests (in tests/).
\end{sphinxadmonition}

\begin{sphinxadmonition}{warning}{Warning:}
This module won’t work properly unless it is executed as root.
\end{sphinxadmonition}
\index{assemble\_lvm\_disk\_info() (in module getdevinfo.linux)@\spxentry{assemble\_lvm\_disk\_info()}\spxextra{in module getdevinfo.linux}}

\begin{fulllineitems}
\phantomsection\label{\detokenize{linux:getdevinfo.linux.assemble_lvm_disk_info}}\pysiglinewithargsret{\sphinxcode{\sphinxupquote{getdevinfo.linux.}}\sphinxbfcode{\sphinxupquote{assemble\_lvm\_disk\_info}}}{\emph{line\_counter}, \emph{testing=False}}{}
Private, implementation detail.

This function is used to assemble LVM disk info into the dictionary.

Like get\_device\_info(), and get\_partition\_info(), it uses some of the
helper functions here.
\begin{description}
\item[{Args:}] \leavevmode\begin{description}
\item[{line\_counter (int):   The line in the output that informtion for a}] \leavevmode
particular logical volume begins.

\end{description}

\item[{Kwargs:}] \leavevmode
testing (bool):       Used during unit tests. Default = False.

\end{description}

Usage:

\begin{sphinxVerbatim}[commandchars=\\\{\}]
\PYG{g+gp}{\PYGZgt{}\PYGZgt{}\PYGZgt{} }\PYG{n}{assemble\PYGZus{}lvm\PYGZus{}disk\PYGZus{}info}\PYG{p}{(}\PYG{o}{\PYGZlt{}}\PYG{n}{anInt}\PYG{o}{\PYGZgt{}}\PYG{p}{)}
\end{sphinxVerbatim}

OR:

\begin{sphinxVerbatim}[commandchars=\\\{\}]
\PYG{g+gp}{\PYGZgt{}\PYGZgt{}\PYGZgt{} }\PYG{n}{assemble\PYGZus{}lvm\PYGZus{}disk\PYGZus{}info}\PYG{p}{(}\PYG{o}{\PYGZlt{}}\PYG{n}{anInt}\PYG{o}{\PYGZgt{}}\PYG{p}{,} \PYG{n}{testing}\PYG{o}{=}\PYG{o}{\PYGZlt{}}\PYG{n}{aBool}\PYG{o}{\PYGZgt{}}\PYG{p}{)}
\end{sphinxVerbatim}

\end{fulllineitems}

\index{compute\_block\_size() (in module getdevinfo.linux)@\spxentry{compute\_block\_size()}\spxextra{in module getdevinfo.linux}}

\begin{fulllineitems}
\phantomsection\label{\detokenize{linux:getdevinfo.linux.compute_block_size}}\pysiglinewithargsret{\sphinxcode{\sphinxupquote{getdevinfo.linux.}}\sphinxbfcode{\sphinxupquote{compute\_block\_size}}}{\emph{stdout}}{}
Private, implementation detail.

Used to process and tidy up the block size output from blockdev.
\begin{description}
\item[{Args:}] \leavevmode
stdout (str):       blockdev’s output.

\item[{Returns:}] \leavevmode
int/None: The block size:
\begin{itemize}
\item {} 
None - Failed!

\item {} 
int  - The block size.

\end{itemize}

\end{description}

Usage:

\begin{sphinxVerbatim}[commandchars=\\\{\}]
\PYG{g+gp}{\PYGZgt{}\PYGZgt{}\PYGZgt{} }\PYG{n}{compute\PYGZus{}block\PYGZus{}size}\PYG{p}{(}\PYG{o}{\PYGZlt{}}\PYG{n}{stdoutFromBlockDev}\PYG{o}{\PYGZgt{}}\PYG{p}{)}
\end{sphinxVerbatim}

\end{fulllineitems}

\index{get\_block\_size() (in module getdevinfo.linux)@\spxentry{get\_block\_size()}\spxextra{in module getdevinfo.linux}}

\begin{fulllineitems}
\phantomsection\label{\detokenize{linux:getdevinfo.linux.get_block_size}}\pysiglinewithargsret{\sphinxcode{\sphinxupquote{getdevinfo.linux.}}\sphinxbfcode{\sphinxupquote{get\_block\_size}}}{\emph{disk}}{}
\sphinxstylestrong{Public}

This function uses the blockdev command to get the block size
of the given device.
\begin{description}
\item[{Args:}] \leavevmode\begin{description}
\item[{disk (str):     The partition/device/logical volume that}] \leavevmode
we want the block size for.

\end{description}

\item[{Returns:}] \leavevmode
int/None. The block size.
\begin{itemize}
\item {} 
None - Failed!

\item {} 
int  - The block size.

\end{itemize}

\end{description}

Usage:

\begin{sphinxVerbatim}[commandchars=\\\{\}]
\PYG{g+gp}{\PYGZgt{}\PYGZgt{}\PYGZgt{} }\PYG{n}{block\PYGZus{}size} \PYG{o}{=} \PYG{n}{get\PYGZus{}block\PYGZus{}size}\PYG{p}{(}\PYG{o}{\PYGZlt{}}\PYG{n}{aDeviceName}\PYG{o}{\PYGZgt{}}\PYG{p}{)}
\end{sphinxVerbatim}

\end{fulllineitems}

\index{get\_boot\_record() (in module getdevinfo.linux)@\spxentry{get\_boot\_record()}\spxextra{in module getdevinfo.linux}}

\begin{fulllineitems}
\phantomsection\label{\detokenize{linux:getdevinfo.linux.get_boot_record}}\pysiglinewithargsret{\sphinxcode{\sphinxupquote{getdevinfo.linux.}}\sphinxbfcode{\sphinxupquote{get\_boot\_record}}}{\emph{disk}}{}
Private, implementation detail.

This function gets the MBR/PBR of a given disk.
\begin{description}
\item[{Args:}] \leavevmode
disk (str):   The name of a partition/device.

\item[{Returns:}] \leavevmode
tuple (string, string). The boot record (raw, any readable strings):
\begin{itemize}
\item {} 
(“Unknown”, “Unknown”)     - Couldn’t read it.

\item {} 
Anything else              - The PBR/MBR and any readable strings therein.

\end{itemize}

\end{description}

Usage:

\begin{sphinxVerbatim}[commandchars=\\\{\}]
\PYG{g+gp}{\PYGZgt{}\PYGZgt{}\PYGZgt{} }\PYG{n}{boot\PYGZus{}record}\PYG{p}{,} \PYG{n}{boot\PYGZus{}record\PYGZus{}strings} \PYG{o}{=} \PYG{n}{get\PYGZus{}boot\PYGZus{}record}\PYG{p}{(}\PYG{o}{\PYGZlt{}}\PYG{n}{aDiskName}\PYG{o}{\PYGZgt{}}\PYG{p}{)}
\end{sphinxVerbatim}

\end{fulllineitems}

\index{get\_capabilities() (in module getdevinfo.linux)@\spxentry{get\_capabilities()}\spxextra{in module getdevinfo.linux}}

\begin{fulllineitems}
\phantomsection\label{\detokenize{linux:getdevinfo.linux.get_capabilities}}\pysiglinewithargsret{\sphinxcode{\sphinxupquote{getdevinfo.linux.}}\sphinxbfcode{\sphinxupquote{get\_capabilities}}}{\emph{node}}{}
Private, implementation detail.

This function gets the capabilities from the structure
generated by parsing lshw’s XML output.
\begin{description}
\item[{Args:}] \leavevmode
node:   Represents a device/partition.

\item[{Returns:}] \leavevmode
list. The capabilities:
\begin{itemize}
\item {} 
{[}{]}            - Couldn’t find them.

\item {} 
Anything else - The capabilities - as unicode strings.

\end{itemize}

\end{description}

Usage:

\begin{sphinxVerbatim}[commandchars=\\\{\}]
\PYG{g+gp}{\PYGZgt{}\PYGZgt{}\PYGZgt{} }\PYG{n}{capabilities} \PYG{o}{=} \PYG{n}{get\PYGZus{}capabilities}\PYG{p}{(}\PYG{o}{\PYGZlt{}}\PYG{n}{aNode}\PYG{o}{\PYGZgt{}}\PYG{p}{)}
\end{sphinxVerbatim}

\end{fulllineitems}

\index{get\_capacity() (in module getdevinfo.linux)@\spxentry{get\_capacity()}\spxextra{in module getdevinfo.linux}}

\begin{fulllineitems}
\phantomsection\label{\detokenize{linux:getdevinfo.linux.get_capacity}}\pysiglinewithargsret{\sphinxcode{\sphinxupquote{getdevinfo.linux.}}\sphinxbfcode{\sphinxupquote{get\_capacity}}}{\emph{node}}{}
Private, implementation detail.

This function gets the capacity from the structure generated
by parsing lshw’s XML output. Also rounds it to a human-
readable form, and returns both sizes.
\begin{description}
\item[{Args:}] \leavevmode
node:   Represents a device/partition.

\item[{Returns:}] \leavevmode
tuple (string, string). The sizes (bytes, human-readable):
\begin{itemize}
\item {} 
(“Unknown”, “Unknown”)     - Couldn’t find them.

\item {} 
Anything else              - The sizes.

\end{itemize}

\end{description}

Usage:

\begin{sphinxVerbatim}[commandchars=\\\{\}]
\PYG{g+gp}{\PYGZgt{}\PYGZgt{}\PYGZgt{} }\PYG{n}{raw\PYGZus{}size}\PYG{p}{,} \PYG{n}{human\PYGZus{}size} \PYG{o}{=} \PYG{n}{get\PYGZus{}capacity}\PYG{p}{(}\PYG{o}{\PYGZlt{}}\PYG{n}{aNode}\PYG{o}{\PYGZgt{}}\PYG{p}{)}
\end{sphinxVerbatim}

\end{fulllineitems}

\index{get\_device\_info() (in module getdevinfo.linux)@\spxentry{get\_device\_info()}\spxextra{in module getdevinfo.linux}}

\begin{fulllineitems}
\phantomsection\label{\detokenize{linux:getdevinfo.linux.get_device_info}}\pysiglinewithargsret{\sphinxcode{\sphinxupquote{getdevinfo.linux.}}\sphinxbfcode{\sphinxupquote{get\_device\_info}}}{\emph{node}}{}
Private, implementation detail.

This function gathers and assembles information for devices (whole disks).
It employs some simple logic and the other functions defined in this
module to do its work.
\begin{description}
\item[{Args:}] \leavevmode\begin{description}
\item[{node:       A “node” representing a device, generated from lshw’s XML}] \leavevmode
output.

\end{description}

\item[{Returns:}] \leavevmode
string.     The name of the device.

\end{description}

Usage:

\begin{sphinxVerbatim}[commandchars=\\\{\}]
\PYG{g+gp}{\PYGZgt{}\PYGZgt{}\PYGZgt{} }\PYG{n}{host\PYGZus{}disk} \PYG{o}{=} \PYG{n}{get\PYGZus{}device\PYGZus{}info}\PYG{p}{(}\PYG{o}{\PYGZlt{}}\PYG{n}{aNode}\PYG{o}{\PYGZgt{}}\PYG{p}{)}
\end{sphinxVerbatim}

\end{fulllineitems}

\index{get\_file\_system() (in module getdevinfo.linux)@\spxentry{get\_file\_system()}\spxextra{in module getdevinfo.linux}}

\begin{fulllineitems}
\phantomsection\label{\detokenize{linux:getdevinfo.linux.get_file_system}}\pysiglinewithargsret{\sphinxcode{\sphinxupquote{getdevinfo.linux.}}\sphinxbfcode{\sphinxupquote{get\_file\_system}}}{\emph{node}}{}
Private, implementation detail.

This function gets the file system from the structure
generated by parsing lshw’s XML output.
\begin{description}
\item[{Args:}] \leavevmode
node:   Represents a device/partition.

\item[{Returns:}] \leavevmode
string. The file system:
\begin{itemize}
\item {} 
“Unknown”     - Couldn’t find it.

\item {} 
Anything else - The file system.

\end{itemize}

\end{description}

Usage:

\begin{sphinxVerbatim}[commandchars=\\\{\}]
\PYG{g+gp}{\PYGZgt{}\PYGZgt{}\PYGZgt{} }\PYG{n}{file\PYGZus{}system} \PYG{o}{=} \PYG{n}{get\PYGZus{}file\PYGZus{}system}\PYG{p}{(}\PYG{o}{\PYGZlt{}}\PYG{n}{aNode}\PYG{o}{\PYGZgt{}}\PYG{p}{)}
\end{sphinxVerbatim}

\end{fulllineitems}

\index{get\_id() (in module getdevinfo.linux)@\spxentry{get\_id()}\spxextra{in module getdevinfo.linux}}

\begin{fulllineitems}
\phantomsection\label{\detokenize{linux:getdevinfo.linux.get_id}}\pysiglinewithargsret{\sphinxcode{\sphinxupquote{getdevinfo.linux.}}\sphinxbfcode{\sphinxupquote{get\_id}}}{\emph{disk}}{}
Private, implementation detail.

This function gets the ID of a given partition or device.
\begin{description}
\item[{Args:}] \leavevmode
disk (str):   The name of a partition/device.

\item[{Returns:}] \leavevmode
string. The ID:
\begin{itemize}
\item {} 
“Unknown”     - Couldn’t find it.

\item {} 
Anything else - The ID.

\end{itemize}

\end{description}

Usage:

\begin{sphinxVerbatim}[commandchars=\\\{\}]
\PYG{g+gp}{\PYGZgt{}\PYGZgt{}\PYGZgt{} }\PYG{n}{disk\PYGZus{}id} \PYG{o}{=} \PYG{n}{get\PYGZus{}id}\PYG{p}{(}\PYG{o}{\PYGZlt{}}\PYG{n}{aDiskName}\PYG{o}{\PYGZgt{}}\PYG{p}{)}
\end{sphinxVerbatim}

\end{fulllineitems}

\index{get\_info() (in module getdevinfo.linux)@\spxentry{get\_info()}\spxextra{in module getdevinfo.linux}}

\begin{fulllineitems}
\phantomsection\label{\detokenize{linux:getdevinfo.linux.get_info}}\pysiglinewithargsret{\sphinxcode{\sphinxupquote{getdevinfo.linux.}}\sphinxbfcode{\sphinxupquote{get\_info}}}{}{}
This function is the Linux-specific way of getting disk information.
It makes use of the lshw, blkid, and lvdisplay commands to gather
information.

It uses the other functions in this module to acheive its work, and
it \sphinxstylestrong{doesn’t} return the disk infomation. Instead, it is left as a
global attribute in this module (DISKINFO).
\begin{description}
\item[{Raises:}] \leavevmode
Nothing, hopefully, but errors have a small chance of propagation
up to here here. Wrap it in a try:, except: block if you are worried.

\end{description}

Usage:

\begin{sphinxVerbatim}[commandchars=\\\{\}]
\PYG{g+gp}{\PYGZgt{}\PYGZgt{}\PYGZgt{} }\PYG{n}{get\PYGZus{}info}\PYG{p}{(}\PYG{p}{)}
\end{sphinxVerbatim}

\end{fulllineitems}

\index{get\_lv\_aliases() (in module getdevinfo.linux)@\spxentry{get\_lv\_aliases()}\spxextra{in module getdevinfo.linux}}

\begin{fulllineitems}
\phantomsection\label{\detokenize{linux:getdevinfo.linux.get_lv_aliases}}\pysiglinewithargsret{\sphinxcode{\sphinxupquote{getdevinfo.linux.}}\sphinxbfcode{\sphinxupquote{get\_lv\_aliases}}}{\emph{line}}{}
Private, implementation detail.

\begin{sphinxadmonition}{note}{Note:}
“name” here means path eg /dev/mapper/fedora/root.
\end{sphinxadmonition}

This function gets the names of a logical volume.
There may be one or more aliases as well as a “default”
name. Find and return all of them.
\begin{description}
\item[{Args:}] \leavevmode
line (int):   The line number where the LV name can be found.

\item[{Returns:}] \leavevmode
tuple (string, list). The aliases (default\_name, all aliases).

\end{description}

Usage:

\begin{sphinxVerbatim}[commandchars=\\\{\}]
\PYG{g+gp}{\PYGZgt{}\PYGZgt{}\PYGZgt{} }\PYG{n}{default\PYGZus{}name}\PYG{p}{,} \PYG{n}{alias\PYGZus{}list} \PYG{o}{=} \PYG{n}{get\PYGZus{}lv\PYGZus{}aliases}\PYG{p}{(}\PYG{o}{\PYGZlt{}}\PYG{n}{anLVName}\PYG{o}{\PYGZgt{}}\PYG{p}{)}
\end{sphinxVerbatim}

\end{fulllineitems}

\index{get\_lv\_and\_vg\_name() (in module getdevinfo.linux)@\spxentry{get\_lv\_and\_vg\_name()}\spxextra{in module getdevinfo.linux}}

\begin{fulllineitems}
\phantomsection\label{\detokenize{linux:getdevinfo.linux.get_lv_and_vg_name}}\pysiglinewithargsret{\sphinxcode{\sphinxupquote{getdevinfo.linux.}}\sphinxbfcode{\sphinxupquote{get\_lv\_and\_vg\_name}}}{\emph{volume}}{}
Private, implementation detail.

\begin{sphinxadmonition}{note}{Note:}
“name” here means the names of the logical volume and
the volume group by themselves. eg volume “root”, in
volume group “fedora.”
\end{sphinxadmonition}

This function gets the name of the logical volume (LV), and the
name of the volume group (VG) it belongs to.
\begin{description}
\item[{Args:}] \leavevmode
volume (str):   The path for a logical volume.

\item[{Returns:}] \leavevmode
tuple (string, string). The VG, and LV name (vg\_name, lv\_name):
\begin{itemize}
\item {} 
(“Unknown”, “Unknown”) - Couldn’t find them.

\item {} 
Anything else          - The VG and LV names.

\end{itemize}

\end{description}

Usage:

\begin{sphinxVerbatim}[commandchars=\\\{\}]
\PYG{g+gp}{\PYGZgt{}\PYGZgt{}\PYGZgt{} }\PYG{n}{vg\PYGZus{}name}\PYG{p}{,} \PYG{n}{lv\PYGZus{}name} \PYG{o}{=} \PYG{n}{get\PYGZus{}lv\PYGZus{}and\PYGZus{}vg\PYGZus{}name}\PYG{p}{(}\PYG{o}{\PYGZlt{}}\PYG{n}{anLVPath}\PYG{o}{\PYGZgt{}}\PYG{p}{)}
\end{sphinxVerbatim}

\end{fulllineitems}

\index{get\_lv\_file\_system() (in module getdevinfo.linux)@\spxentry{get\_lv\_file\_system()}\spxextra{in module getdevinfo.linux}}

\begin{fulllineitems}
\phantomsection\label{\detokenize{linux:getdevinfo.linux.get_lv_file_system}}\pysiglinewithargsret{\sphinxcode{\sphinxupquote{getdevinfo.linux.}}\sphinxbfcode{\sphinxupquote{get\_lv\_file\_system}}}{\emph{disk}}{}
Private, implementation detail.

This function gets the file system of a logical volume.
\begin{description}
\item[{Args:}] \leavevmode
disk (str):   The name of a logical volume.

\item[{Returns:}] \leavevmode
string. The file system.

\end{description}

Usage:

\begin{sphinxVerbatim}[commandchars=\\\{\}]
\PYG{g+gp}{\PYGZgt{}\PYGZgt{}\PYGZgt{} }\PYG{n}{file\PYGZus{}system} \PYG{o}{=} \PYG{n}{get\PYGZus{}lv\PYGZus{}file\PYGZus{}system}\PYG{p}{(}\PYG{o}{\PYGZlt{}}\PYG{n}{anLVName}\PYG{o}{\PYGZgt{}}\PYG{p}{)}
\end{sphinxVerbatim}

\end{fulllineitems}

\index{get\_partition\_info() (in module getdevinfo.linux)@\spxentry{get\_partition\_info()}\spxextra{in module getdevinfo.linux}}

\begin{fulllineitems}
\phantomsection\label{\detokenize{linux:getdevinfo.linux.get_partition_info}}\pysiglinewithargsret{\sphinxcode{\sphinxupquote{getdevinfo.linux.}}\sphinxbfcode{\sphinxupquote{get\_partition\_info}}}{\emph{subnode}, \emph{host\_disk}}{}
Private, implementation detail.

This function gathers and assembles information for partitions.
It employs some simple logic and the other functions defined in this
module to do its work.
\begin{description}
\item[{Args:}] \leavevmode\begin{description}
\item[{subnode:            A “node” representing a partition, generated}] \leavevmode
from lshw’s XML output.

\item[{host\_disk (str):    The “parent” or “host” device. eg: for}] \leavevmode
/dev/sda1, the host disk would be /dev/sda.
Used to organise everything nicely in the
disk info dictionary.

\end{description}

\item[{Returns:}] \leavevmode
string.     The name of the partition.

\end{description}

Usage:

\begin{sphinxVerbatim}[commandchars=\\\{\}]
\PYG{g+gp}{\PYGZgt{}\PYGZgt{}\PYGZgt{} }\PYG{n}{volume} \PYG{o}{=} \PYG{n}{get\PYGZus{}device\PYGZus{}info}\PYG{p}{(}\PYG{o}{\PYGZlt{}}\PYG{n}{aNode}\PYG{o}{\PYGZgt{}}\PYG{p}{)}
\end{sphinxVerbatim}

\end{fulllineitems}

\index{get\_partitioning() (in module getdevinfo.linux)@\spxentry{get\_partitioning()}\spxextra{in module getdevinfo.linux}}

\begin{fulllineitems}
\phantomsection\label{\detokenize{linux:getdevinfo.linux.get_partitioning}}\pysiglinewithargsret{\sphinxcode{\sphinxupquote{getdevinfo.linux.}}\sphinxbfcode{\sphinxupquote{get\_partitioning}}}{\emph{disk}}{}
Private, implementation detail.

This function gets the partition scheme from the
structure generated by parsing lshw’s XML output.
\begin{description}
\item[{Args:}] \leavevmode\begin{description}
\item[{disk (str):   The name of a device/partition in}] \leavevmode
the disk info dictionary.

\end{description}

\item[{Returns:}] \leavevmode
string (str). The partition scheme:
\begin{itemize}
\item {} 
“Unknown”     - Couldn’t find it.

\item {} \begin{description}
\item[{“mbr”         - Old-style MBR partitioning}] \leavevmode
for BIOS systems.

\end{description}

\item {} 
“gpt”         - New-style GPT partitioning.

\end{itemize}

\end{description}

Usage:

\begin{sphinxVerbatim}[commandchars=\\\{\}]
\PYG{g+gp}{\PYGZgt{}\PYGZgt{}\PYGZgt{} }\PYG{n}{partitioning} \PYG{o}{=} \PYG{n}{get\PYGZus{}partitioning}\PYG{p}{(}\PYG{o}{\PYGZlt{}}\PYG{n}{aDiskName}\PYG{o}{\PYGZgt{}}\PYG{p}{)}
\end{sphinxVerbatim}

\end{fulllineitems}

\index{get\_product() (in module getdevinfo.linux)@\spxentry{get\_product()}\spxextra{in module getdevinfo.linux}}

\begin{fulllineitems}
\phantomsection\label{\detokenize{linux:getdevinfo.linux.get_product}}\pysiglinewithargsret{\sphinxcode{\sphinxupquote{getdevinfo.linux.}}\sphinxbfcode{\sphinxupquote{get\_product}}}{\emph{node}}{}
Private, implementation detail.

This function gets the product from the structure generated
by parsing lshw’s XML output.
\begin{description}
\item[{Args:}] \leavevmode
node:   Represents a device/partition.

\item[{Returns:}] \leavevmode
string. The product:
\begin{itemize}
\item {} 
“Unknown”     - Couldn’t find it.

\item {} 
Anything else - The product.

\end{itemize}

\end{description}

Usage:

\begin{sphinxVerbatim}[commandchars=\\\{\}]
\PYG{g+gp}{\PYGZgt{}\PYGZgt{}\PYGZgt{} }\PYG{n}{product} \PYG{o}{=} \PYG{n}{get\PYGZus{}product}\PYG{p}{(}\PYG{o}{\PYGZlt{}}\PYG{n}{aNode}\PYG{o}{\PYGZgt{}}\PYG{p}{)}
\end{sphinxVerbatim}

\end{fulllineitems}

\index{get\_uuid() (in module getdevinfo.linux)@\spxentry{get\_uuid()}\spxextra{in module getdevinfo.linux}}

\begin{fulllineitems}
\phantomsection\label{\detokenize{linux:getdevinfo.linux.get_uuid}}\pysiglinewithargsret{\sphinxcode{\sphinxupquote{getdevinfo.linux.}}\sphinxbfcode{\sphinxupquote{get\_uuid}}}{\emph{disk}}{}
Private, implementation detail.

This function gets the UUID of a given partition.
\begin{description}
\item[{Args:}] \leavevmode
disk (str):   The name of a \sphinxstylestrong{partition}.

\item[{Returns:}] \leavevmode
string. The UUID:
\begin{itemize}
\item {} 
“Unknown”     - Couldn’t find it.

\item {} 
Anything else - The UUID.

\end{itemize}

\end{description}

Usage:

\begin{sphinxVerbatim}[commandchars=\\\{\}]
\PYG{g+gp}{\PYGZgt{}\PYGZgt{}\PYGZgt{} }\PYG{n}{uuid} \PYG{o}{=} \PYG{n}{get\PYGZus{}uuid}\PYG{p}{(}\PYG{o}{\PYGZlt{}}\PYG{n}{aPartitionName}\PYG{o}{\PYGZgt{}}\PYG{p}{)}
\end{sphinxVerbatim}

\end{fulllineitems}

\index{get\_vendor() (in module getdevinfo.linux)@\spxentry{get\_vendor()}\spxextra{in module getdevinfo.linux}}

\begin{fulllineitems}
\phantomsection\label{\detokenize{linux:getdevinfo.linux.get_vendor}}\pysiglinewithargsret{\sphinxcode{\sphinxupquote{getdevinfo.linux.}}\sphinxbfcode{\sphinxupquote{get\_vendor}}}{\emph{node}}{}
Private, implementation detail.

This function gets the vendor from the structure generated
by parsing lshw’s XML output.
\begin{description}
\item[{Args:}] \leavevmode
node:   Represents a device/partition.

\item[{Returns:}] \leavevmode
string. The vendor:
\begin{itemize}
\item {} 
“Unknown”     - Couldn’t find it.

\item {} 
Anything else - The vendor.

\end{itemize}

\end{description}

Usage:

\begin{sphinxVerbatim}[commandchars=\\\{\}]
\PYG{g+gp}{\PYGZgt{}\PYGZgt{}\PYGZgt{} }\PYG{n}{vendor} \PYG{o}{=} \PYG{n}{get\PYGZus{}vendor}\PYG{p}{(}\PYG{o}{\PYGZlt{}}\PYG{n}{aNode}\PYG{o}{\PYGZgt{}}\PYG{p}{)}
\end{sphinxVerbatim}

\end{fulllineitems}

\index{parse\_lsblk\_output() (in module getdevinfo.linux)@\spxentry{parse\_lsblk\_output()}\spxextra{in module getdevinfo.linux}}

\begin{fulllineitems}
\phantomsection\label{\detokenize{linux:getdevinfo.linux.parse_lsblk_output}}\pysiglinewithargsret{\sphinxcode{\sphinxupquote{getdevinfo.linux.}}\sphinxbfcode{\sphinxupquote{parse\_lsblk\_output}}}{}{}
Private, implementation detail.

This function is used to get NVME disk information from the output of lsblk.

\begin{sphinxadmonition}{note}{Note:}
This will only remain here until lshw adds support for NVME disk detection -
this is a temporary fix.
\end{sphinxadmonition}

Usage:

\begin{sphinxVerbatim}[commandchars=\\\{\}]
\PYG{g+gp}{\PYGZgt{}\PYGZgt{}\PYGZgt{} }\PYG{n}{parse\PYGZus{}lsblk\PYGZus{}output}\PYG{p}{(}\PYG{p}{)}
\end{sphinxVerbatim}

\end{fulllineitems}

\index{parse\_lvm\_output() (in module getdevinfo.linux)@\spxentry{parse\_lvm\_output()}\spxextra{in module getdevinfo.linux}}

\begin{fulllineitems}
\phantomsection\label{\detokenize{linux:getdevinfo.linux.parse_lvm_output}}\pysiglinewithargsret{\sphinxcode{\sphinxupquote{getdevinfo.linux.}}\sphinxbfcode{\sphinxupquote{parse\_lvm\_output}}}{\emph{testing=False}}{}
Private, implementation detail.

This function is used to get LVM partition information from the
output of lvdisplay \textendash{}maps.
\begin{description}
\item[{Kwargs:}] \leavevmode
testing (bool):     Used during unit tests. Default = False.

\end{description}

Usage:

\begin{sphinxVerbatim}[commandchars=\\\{\}]
\PYG{g+gp}{\PYGZgt{}\PYGZgt{}\PYGZgt{} }\PYG{n}{parse\PYGZus{}lvm\PYGZus{}output}\PYG{p}{(}\PYG{p}{)}
\end{sphinxVerbatim}

OR:

\begin{sphinxVerbatim}[commandchars=\\\{\}]
\PYG{g+gp}{\PYGZgt{}\PYGZgt{}\PYGZgt{} }\PYG{n}{parse\PYGZus{}lvm\PYGZus{}output}\PYG{p}{(}\PYG{n}{testing}\PYG{o}{=}\PYG{o}{\PYGZlt{}}\PYG{n}{aBool}\PYG{o}{\PYGZgt{}}\PYG{p}{)}
\end{sphinxVerbatim}

\end{fulllineitems}



\chapter{Documentation for the cygwin module}
\label{\detokenize{cygwin:module-getdevinfo.cygwin}}\label{\detokenize{cygwin:documentation-for-the-cygwin-module}}\label{\detokenize{cygwin::doc}}\index{getdevinfo.cygwin (module)@\spxentry{getdevinfo.cygwin}\spxextra{module}}
This is the part of the package that contains the tools and information
getters for Windows/Cygwin. This would normally be called from the getdevinfo
module, but you can call it directly if you like.

\begin{sphinxadmonition}{note}{Note:}
You can import this submodule directly, but it might result
in strange behaviour, or not work on your platform if you
import the wrong one. That is not how the package is intended
to be used.
\end{sphinxadmonition}

\begin{sphinxadmonition}{warning}{Warning:}
Feel free to experiment, but be aware that you may be able to
cause crashes, exceptions, and generally weird situations by calling
these methods directly if you get it wrong. A good place to
look if you’re interested in this is the unit tests (in tests/).
\end{sphinxadmonition}

\begin{sphinxadmonition}{warning}{Warning:}
This module won’t work properly unless it is executed as root.
\end{sphinxadmonition}
\index{compute\_block\_size() (in module getdevinfo.cygwin)@\spxentry{compute\_block\_size()}\spxextra{in module getdevinfo.cygwin}}

\begin{fulllineitems}
\phantomsection\label{\detokenize{cygwin:getdevinfo.cygwin.compute_block_size}}\pysiglinewithargsret{\sphinxcode{\sphinxupquote{getdevinfo.cygwin.}}\sphinxbfcode{\sphinxupquote{compute\_block\_size}}}{\emph{stdout}}{}
Private, implementation detail.

Used to process and tidy up the block size output from smartctl.
\begin{description}
\item[{Args:}] \leavevmode
stdout (str):       The block size.

\item[{Returns:}] \leavevmode
int/None: The block size:
\begin{itemize}
\item {} 
None - Failed!

\item {} 
int  - The block size.

\end{itemize}

\end{description}

Usage:

\begin{sphinxVerbatim}[commandchars=\\\{\}]
\PYG{g+gp}{\PYGZgt{}\PYGZgt{}\PYGZgt{} }\PYG{n}{compute\PYGZus{}block\PYGZus{}size}\PYG{p}{(}\PYG{o}{\PYGZlt{}}\PYG{n}{stdoutFromBlockDev}\PYG{o}{\PYGZgt{}}\PYG{p}{)}
\end{sphinxVerbatim}

\end{fulllineitems}

\index{get\_block\_size() (in module getdevinfo.cygwin)@\spxentry{get\_block\_size()}\spxextra{in module getdevinfo.cygwin}}

\begin{fulllineitems}
\phantomsection\label{\detokenize{cygwin:getdevinfo.cygwin.get_block_size}}\pysiglinewithargsret{\sphinxcode{\sphinxupquote{getdevinfo.cygwin.}}\sphinxbfcode{\sphinxupquote{get\_block\_size}}}{\emph{disk}}{}
\sphinxstylestrong{Public}

This function uses the smartctl command to get the block size
of the given device.
\begin{description}
\item[{Args:}] \leavevmode\begin{description}
\item[{disk (str):     The partition/device/logical volume that}] \leavevmode
we want the block size for.

\end{description}

\item[{Returns:}] \leavevmode
int/None. The block size.
\begin{itemize}
\item {} 
None - Failed!

\item {} 
int  - The block size.

\end{itemize}

\end{description}

Usage:

\begin{sphinxVerbatim}[commandchars=\\\{\}]
\PYG{g+gp}{\PYGZgt{}\PYGZgt{}\PYGZgt{} }\PYG{n}{block\PYGZus{}size} \PYG{o}{=} \PYG{n}{get\PYGZus{}block\PYGZus{}size}\PYG{p}{(}\PYG{o}{\PYGZlt{}}\PYG{n}{aDeviceName}\PYG{o}{\PYGZgt{}}\PYG{p}{)}
\end{sphinxVerbatim}

\end{fulllineitems}

\index{get\_boot\_record() (in module getdevinfo.cygwin)@\spxentry{get\_boot\_record()}\spxextra{in module getdevinfo.cygwin}}

\begin{fulllineitems}
\phantomsection\label{\detokenize{cygwin:getdevinfo.cygwin.get_boot_record}}\pysiglinewithargsret{\sphinxcode{\sphinxupquote{getdevinfo.cygwin.}}\sphinxbfcode{\sphinxupquote{get\_boot\_record}}}{\emph{disk}}{}
Private, implementation detail.

This function gets the MBR/PBR of a given disk.
\begin{description}
\item[{Args:}] \leavevmode
disk (str):   The name of a partition/device.

\item[{Returns:}] \leavevmode
tuple (string, string). The boot record (raw, any readable strings):
\begin{itemize}
\item {} 
(“Unknown”, “Unknown”)     - Couldn’t read it.

\item {} 
Anything else              - The PBR/MBR and any readable strings therein.

\end{itemize}

\end{description}

Usage:

\begin{sphinxVerbatim}[commandchars=\\\{\}]
\PYG{g+gp}{\PYGZgt{}\PYGZgt{}\PYGZgt{} }\PYG{n}{boot\PYGZus{}record}\PYG{p}{,} \PYG{n}{boot\PYGZus{}record\PYGZus{}strings} \PYG{o}{=} \PYG{n}{get\PYGZus{}boot\PYGZus{}record}\PYG{p}{(}\PYG{o}{\PYGZlt{}}\PYG{n}{aDiskName}\PYG{o}{\PYGZgt{}}\PYG{p}{)}
\end{sphinxVerbatim}

\end{fulllineitems}

\index{get\_capabilities() (in module getdevinfo.cygwin)@\spxentry{get\_capabilities()}\spxextra{in module getdevinfo.cygwin}}

\begin{fulllineitems}
\phantomsection\label{\detokenize{cygwin:getdevinfo.cygwin.get_capabilities}}\pysiglinewithargsret{\sphinxcode{\sphinxupquote{getdevinfo.cygwin.}}\sphinxbfcode{\sphinxupquote{get\_capabilities}}}{\emph{disk}}{}
Private, implementation detail.

This function gets the capabilities from the structure generated
by parsing smartctl’s output.

\begin{sphinxadmonition}{warning}{Warning:}
Not yet implemented on Cygwin, returns empty list.
\end{sphinxadmonition}
\begin{description}
\item[{Args:}] \leavevmode
data (dict):   Parsed JSON from smartctl.

\item[{Returns:}] \leavevmode
list. The capabilities:
\begin{itemize}
\item {} 
{[}{]}            - Couldn’t find them.

\item {} 
Anything else - The capabilities - as unicode strings.

\end{itemize}

\end{description}

Usage:

\begin{sphinxVerbatim}[commandchars=\\\{\}]
\PYG{g+gp}{\PYGZgt{}\PYGZgt{}\PYGZgt{} }\PYG{n}{capabilities} \PYG{o}{=} \PYG{n}{get\PYGZus{}capabilities}\PYG{p}{(}\PYG{o}{\PYGZlt{}}\PYG{n}{aNode}\PYG{o}{\PYGZgt{}}\PYG{p}{)}
\end{sphinxVerbatim}

\end{fulllineitems}

\index{get\_capacity() (in module getdevinfo.cygwin)@\spxentry{get\_capacity()}\spxextra{in module getdevinfo.cygwin}}

\begin{fulllineitems}
\phantomsection\label{\detokenize{cygwin:getdevinfo.cygwin.get_capacity}}\pysiglinewithargsret{\sphinxcode{\sphinxupquote{getdevinfo.cygwin.}}\sphinxbfcode{\sphinxupquote{get\_capacity}}}{\emph{data}}{}
Private, implementation detail.

This function gets the vendor from the structure generated
by parsing smartctl’s output. Also rounds it to a human-readable
form, and returns both pieces of data.
\begin{description}
\item[{Args:}] \leavevmode
data:   Parsed JSON from smartctl.

\item[{Returns:}] \leavevmode
tuple (string, string). The sizes (bytes, human-readable):
\begin{itemize}
\item {} 
(“Unknown”, “Unknown”)     - Couldn’t find them.

\item {} 
Anything else              - The sizes.

\end{itemize}

\end{description}

Usage:

\begin{sphinxVerbatim}[commandchars=\\\{\}]
\PYG{g+gp}{\PYGZgt{}\PYGZgt{}\PYGZgt{} }\PYG{n}{raw\PYGZus{}size}\PYG{p}{,} \PYG{n}{human\PYGZus{}size} \PYG{o}{=} \PYG{n}{get\PYGZus{}capacity}\PYG{p}{(}\PYG{o}{\PYGZlt{}}\PYG{n}{smartctl}\PYG{o}{\PYGZhy{}}\PYG{n}{data}\PYG{o}{\PYGZgt{}}\PYG{p}{)}
\end{sphinxVerbatim}

\end{fulllineitems}

\index{get\_description() (in module getdevinfo.cygwin)@\spxentry{get\_description()}\spxextra{in module getdevinfo.cygwin}}

\begin{fulllineitems}
\phantomsection\label{\detokenize{cygwin:getdevinfo.cygwin.get_description}}\pysiglinewithargsret{\sphinxcode{\sphinxupquote{getdevinfo.cygwin.}}\sphinxbfcode{\sphinxupquote{get\_description}}}{\emph{data}, \emph{disk}}{}
Private, implementation detail.

This function creates a description from the structure generated
by parsing smartctl’s output.
\begin{description}
\item[{Args:}] \leavevmode
data:         Parsed JSON from smartctl.
disk (str):   Name of a device/partition.

\item[{Returns:}] \leavevmode\begin{description}
\item[{string. The description: This may contain various bits of info, or not,}] \leavevmode
depending on what macOS knows about the disk.

\end{description}

\end{description}

Usage:

\begin{sphinxVerbatim}[commandchars=\\\{\}]
\PYG{g+gp}{\PYGZgt{}\PYGZgt{}\PYGZgt{} }\PYG{n}{description} \PYG{o}{=} \PYG{n}{get\PYGZus{}description}\PYG{p}{(}\PYG{o}{\PYGZlt{}}\PYG{n}{smartctl}\PYG{o}{\PYGZhy{}}\PYG{n}{data}\PYG{o}{\PYGZgt{}}\PYG{p}{,} \PYG{o}{\PYGZlt{}}\PYG{n}{aDisk}\PYG{o}{\PYGZgt{}}\PYG{p}{)}
\end{sphinxVerbatim}

\end{fulllineitems}

\index{get\_device\_info() (in module getdevinfo.cygwin)@\spxentry{get\_device\_info()}\spxextra{in module getdevinfo.cygwin}}

\begin{fulllineitems}
\phantomsection\label{\detokenize{cygwin:getdevinfo.cygwin.get_device_info}}\pysiglinewithargsret{\sphinxcode{\sphinxupquote{getdevinfo.cygwin.}}\sphinxbfcode{\sphinxupquote{get\_device\_info}}}{\emph{host\_disk}}{}
Private, implementation detail.

This function gathers and assembles information for devices (whole disks).
It employs some simple logic and the other functions defined in this
module to do its work.

\begin{sphinxadmonition}{note}{Note:}
Functionality not yet complete.
\end{sphinxadmonition}
\begin{description}
\item[{Args:}] \leavevmode
host\_disk:  The name of the device.

\item[{Returns:}] \leavevmode
string.     The name of the device.

\end{description}

Usage:

\begin{sphinxVerbatim}[commandchars=\\\{\}]
\PYG{g+gp}{\PYGZgt{}\PYGZgt{}\PYGZgt{} }\PYG{n}{host\PYGZus{}disk} \PYG{o}{=} \PYG{n}{get\PYGZus{}device\PYGZus{}info}\PYG{p}{(}\PYG{o}{\PYGZlt{}}\PYG{n}{aNode}\PYG{o}{\PYGZgt{}}\PYG{p}{)}
\end{sphinxVerbatim}

\end{fulllineitems}

\index{get\_file\_system() (in module getdevinfo.cygwin)@\spxentry{get\_file\_system()}\spxextra{in module getdevinfo.cygwin}}

\begin{fulllineitems}
\phantomsection\label{\detokenize{cygwin:getdevinfo.cygwin.get_file_system}}\pysiglinewithargsret{\sphinxcode{\sphinxupquote{getdevinfo.cygwin.}}\sphinxbfcode{\sphinxupquote{get\_file\_system}}}{\emph{output}}{}
Private, implementation detail.

This function gets the file system from
blkid’s output.
\begin{description}
\item[{Args:}] \leavevmode
output (list):   Output from blkid.

\item[{Returns:}] \leavevmode
string. The file system:
\begin{itemize}
\item {} 
“Unknown”     - Couldn’t find it.

\item {} 
Anything else - The file system.

\end{itemize}

\end{description}

Usage:

\begin{sphinxVerbatim}[commandchars=\\\{\}]
\PYG{g+gp}{\PYGZgt{}\PYGZgt{}\PYGZgt{} }\PYG{n}{file\PYGZus{}system} \PYG{o}{=} \PYG{n}{get\PYGZus{}file\PYGZus{}system}\PYG{p}{(}\PYG{o}{\PYGZlt{}}\PYG{n}{blkid}\PYG{o}{\PYGZhy{}}\PYG{n}{output}\PYG{o}{\PYGZgt{}}\PYG{p}{)}
\end{sphinxVerbatim}

\end{fulllineitems}

\index{get\_id() (in module getdevinfo.cygwin)@\spxentry{get\_id()}\spxextra{in module getdevinfo.cygwin}}

\begin{fulllineitems}
\phantomsection\label{\detokenize{cygwin:getdevinfo.cygwin.get_id}}\pysiglinewithargsret{\sphinxcode{\sphinxupquote{getdevinfo.cygwin.}}\sphinxbfcode{\sphinxupquote{get\_id}}}{\emph{disk}}{}
Private, implementation detail.

This function gets the ID of a given partition or device.

\begin{sphinxadmonition}{warning}{Warning:}
Not yet implemented on Cygwin.
\end{sphinxadmonition}
\begin{description}
\item[{Args:}] \leavevmode
disk (str):   The name of a partition/device.

\item[{Returns:}] \leavevmode
string. The ID:
\begin{itemize}
\item {} 
“Unknown”     - Couldn’t find it.

\item {} 
Anything else - The ID.

\end{itemize}

\end{description}

Usage:

\begin{sphinxVerbatim}[commandchars=\\\{\}]
\PYG{g+gp}{\PYGZgt{}\PYGZgt{}\PYGZgt{} }\PYG{n}{disk\PYGZus{}id} \PYG{o}{=} \PYG{n}{get\PYGZus{}id}\PYG{p}{(}\PYG{o}{\PYGZlt{}}\PYG{n}{aDiskName}\PYG{o}{\PYGZgt{}}\PYG{p}{)}
\end{sphinxVerbatim}

\end{fulllineitems}

\index{get\_info() (in module getdevinfo.cygwin)@\spxentry{get\_info()}\spxextra{in module getdevinfo.cygwin}}

\begin{fulllineitems}
\phantomsection\label{\detokenize{cygwin:getdevinfo.cygwin.get_info}}\pysiglinewithargsret{\sphinxcode{\sphinxupquote{getdevinfo.cygwin.}}\sphinxbfcode{\sphinxupquote{get\_info}}}{}{}
This function is the Cygwin-specific way of getting disk information.
It makes use of the smartctl and blkid commands to gather
information.

It uses the other functions in this module to acheive its work, and
it \sphinxstylestrong{doesn’t} return the disk infomation. Instead, it is left as a
global attribute in this module (DISKINFO).
\begin{description}
\item[{Raises:}] \leavevmode
Nothing, hopefully, but errors have a small chance of propagation
up to here here. Wrap it in a try:, except: block if you are worried.

\end{description}

Usage:

\begin{sphinxVerbatim}[commandchars=\\\{\}]
\PYG{g+gp}{\PYGZgt{}\PYGZgt{}\PYGZgt{} }\PYG{n}{get\PYGZus{}info}\PYG{p}{(}\PYG{p}{)}
\end{sphinxVerbatim}

\end{fulllineitems}

\index{get\_partition\_info() (in module getdevinfo.cygwin)@\spxentry{get\_partition\_info()}\spxextra{in module getdevinfo.cygwin}}

\begin{fulllineitems}
\phantomsection\label{\detokenize{cygwin:getdevinfo.cygwin.get_partition_info}}\pysiglinewithargsret{\sphinxcode{\sphinxupquote{getdevinfo.cygwin.}}\sphinxbfcode{\sphinxupquote{get\_partition\_info}}}{\emph{subnode}, \emph{host\_disk}}{}
Private, implementation detail.

This function gathers and assembles information for partitions.
It employs some simple logic and the other functions defined in this
module to do its work.

\begin{sphinxadmonition}{warning}{Warning:}
Not yet implemented on Cygwin - returns None.
\end{sphinxadmonition}
\begin{description}
\item[{Args:}] \leavevmode\begin{description}
\item[{subnode:            A “node” representing a partition, generated}] \leavevmode
from lshw’s XML output.

\item[{host\_disk (str):    The “parent” or “host” device. eg: for}] \leavevmode
/dev/sda1, the host disk would be /dev/sda.
Used to organise everything nicely in the
disk info dictionary.

\end{description}

\item[{Returns:}] \leavevmode
string.     The name of the partition.

\end{description}

Usage:

\begin{sphinxVerbatim}[commandchars=\\\{\}]
\PYG{g+gp}{\PYGZgt{}\PYGZgt{}\PYGZgt{} }\PYG{n}{volume} \PYG{o}{=} \PYG{n}{get\PYGZus{}device\PYGZus{}info}\PYG{p}{(}\PYG{o}{\PYGZlt{}}\PYG{n}{aNode}\PYG{o}{\PYGZgt{}}\PYG{p}{)}
\end{sphinxVerbatim}

\end{fulllineitems}

\index{get\_partitioning() (in module getdevinfo.cygwin)@\spxentry{get\_partitioning()}\spxextra{in module getdevinfo.cygwin}}

\begin{fulllineitems}
\phantomsection\label{\detokenize{cygwin:getdevinfo.cygwin.get_partitioning}}\pysiglinewithargsret{\sphinxcode{\sphinxupquote{getdevinfo.cygwin.}}\sphinxbfcode{\sphinxupquote{get\_partitioning}}}{\emph{output}}{}
Private, implementation detail.

This function gets the partition scheme from
blkid’s output.
\begin{description}
\item[{Args:}] \leavevmode
output (list):   Output from blkid.

\item[{Returns:}] \leavevmode
string (str). The partition scheme:
\begin{itemize}
\item {} 
“Unknown”     - Couldn’t find it.

\item {} \begin{description}
\item[{“mbr”         - Old-style MBR partitioning}] \leavevmode
for BIOS systems.

\end{description}

\item {} 
“gpt”         - New-style GPT partitioning.

\end{itemize}

\end{description}

Usage:

\begin{sphinxVerbatim}[commandchars=\\\{\}]
\PYG{g+gp}{\PYGZgt{}\PYGZgt{}\PYGZgt{} }\PYG{n}{partitioning} \PYG{o}{=} \PYG{n}{get\PYGZus{}partitioning}\PYG{p}{(}\PYG{o}{\PYGZlt{}}\PYG{n}{blkid}\PYG{o}{\PYGZhy{}}\PYG{n}{output}\PYG{o}{\PYGZgt{}}\PYG{p}{)}
\end{sphinxVerbatim}

\end{fulllineitems}

\index{get\_product() (in module getdevinfo.cygwin)@\spxentry{get\_product()}\spxextra{in module getdevinfo.cygwin}}

\begin{fulllineitems}
\phantomsection\label{\detokenize{cygwin:getdevinfo.cygwin.get_product}}\pysiglinewithargsret{\sphinxcode{\sphinxupquote{getdevinfo.cygwin.}}\sphinxbfcode{\sphinxupquote{get\_product}}}{\emph{data}}{}
Private, implementation detail.

This function gets the product from the structure generated
by parsing smartctl’s output.
\begin{description}
\item[{Args:}] \leavevmode
data:   Parsed JSON from smartctl.

\item[{Returns:}] \leavevmode
string. The product:
\begin{itemize}
\item {} 
“Unknown”     - Couldn’t find it.

\item {} 
Anything else - The product.

\end{itemize}

\end{description}

Usage:

\begin{sphinxVerbatim}[commandchars=\\\{\}]
\PYG{g+gp}{\PYGZgt{}\PYGZgt{}\PYGZgt{} }\PYG{n}{product} \PYG{o}{=} \PYG{n}{get\PYGZus{}product}\PYG{p}{(}\PYG{o}{\PYGZlt{}}\PYG{n}{smartctl}\PYG{o}{\PYGZhy{}}\PYG{n}{data}\PYG{o}{\PYGZgt{}}\PYG{p}{)}
\end{sphinxVerbatim}

\end{fulllineitems}

\index{get\_uuid() (in module getdevinfo.cygwin)@\spxentry{get\_uuid()}\spxextra{in module getdevinfo.cygwin}}

\begin{fulllineitems}
\phantomsection\label{\detokenize{cygwin:getdevinfo.cygwin.get_uuid}}\pysiglinewithargsret{\sphinxcode{\sphinxupquote{getdevinfo.cygwin.}}\sphinxbfcode{\sphinxupquote{get\_uuid}}}{\emph{output}}{}
Private, implementation detail.

This function gets the partition scheme from
blkid’s output.
\begin{description}
\item[{Args:}] \leavevmode
output (list):   Output from blkid.

\item[{Returns:}] \leavevmode
string. The UUID:
\begin{itemize}
\item {} 
“Unknown”     - Couldn’t find it.

\item {} 
Anything else - The UUID.

\end{itemize}

\end{description}

Usage:

\begin{sphinxVerbatim}[commandchars=\\\{\}]
\PYG{g+gp}{\PYGZgt{}\PYGZgt{}\PYGZgt{} }\PYG{n}{uuid} \PYG{o}{=} \PYG{n}{get\PYGZus{}uuid}\PYG{p}{(}\PYG{o}{\PYGZlt{}}\PYG{n}{blkid}\PYG{o}{\PYGZhy{}}\PYG{n}{output}\PYG{o}{\PYGZgt{}}\PYG{p}{)}
\end{sphinxVerbatim}

\end{fulllineitems}

\index{get\_vendor() (in module getdevinfo.cygwin)@\spxentry{get\_vendor()}\spxextra{in module getdevinfo.cygwin}}

\begin{fulllineitems}
\phantomsection\label{\detokenize{cygwin:getdevinfo.cygwin.get_vendor}}\pysiglinewithargsret{\sphinxcode{\sphinxupquote{getdevinfo.cygwin.}}\sphinxbfcode{\sphinxupquote{get\_vendor}}}{\emph{data}}{}
Private, implementation detail.

This function gets the vendor from the structure generated
by parsing smartctl’s output.
\begin{description}
\item[{Args:}] \leavevmode
data:   Parsed JSON from smartctl.

\item[{Returns:}] \leavevmode
string. The vendor:
\begin{itemize}
\item {} 
“Unknown”     - Couldn’t find it.

\item {} 
Anything else - The vendor.

\end{itemize}

\end{description}

Usage:

\begin{sphinxVerbatim}[commandchars=\\\{\}]
\PYG{g+gp}{\PYGZgt{}\PYGZgt{}\PYGZgt{} }\PYG{n}{vendor} \PYG{o}{=} \PYG{n}{get\PYGZus{}vendor}\PYG{p}{(}\PYG{o}{\PYGZlt{}}\PYG{n}{smartctl}\PYG{o}{\PYGZhy{}}\PYG{n}{data}\PYG{o}{\PYGZgt{}}\PYG{p}{)}
\end{sphinxVerbatim}

\end{fulllineitems}



\chapter{Documentation for the macos module}
\label{\detokenize{macos:module-getdevinfo.macos}}\label{\detokenize{macos:documentation-for-the-macos-module}}\label{\detokenize{macos::doc}}\index{getdevinfo.macos (module)@\spxentry{getdevinfo.macos}\spxextra{module}}
This is the part of the package that contains the tools and information
getters for macOS. This would normally be called from the getdevinfo
module, but you can call it directly if you like.

\begin{sphinxadmonition}{note}{Note:}
You can import this submodule directly, but it might result
in strange behaviour, or not work on your platform if you
import the wrong one. That is not how the package is intended
to be used, except if you want to use the get\_block\_size()
function to get a block size, as documented below.
\end{sphinxadmonition}

\begin{sphinxadmonition}{warning}{Warning:}
Feel free to experiment, but be aware that you may be able to cause
crashes, exceptions, and generally weird situations by calling
these methods directly if you get it wrong. A good place to
look if you’re interested in this is the unit tests (in tests/).
\end{sphinxadmonition}

\begin{sphinxadmonition}{warning}{Warning:}
This module won’t work properly unless it is executed as root.
\end{sphinxadmonition}
\index{compute\_block\_size() (in module getdevinfo.macos)@\spxentry{compute\_block\_size()}\spxextra{in module getdevinfo.macos}}

\begin{fulllineitems}
\phantomsection\label{\detokenize{macos:getdevinfo.macos.compute_block_size}}\pysiglinewithargsret{\sphinxcode{\sphinxupquote{getdevinfo.macos.}}\sphinxbfcode{\sphinxupquote{compute\_block\_size}}}{\emph{disk}, \emph{stdout}}{}
Private, implementation detail.

Used to process and tidy up the block size output from diskutil info.
\begin{description}
\item[{Args:}] \leavevmode
stdout (str):       diskutil info’s output.

\item[{Returns:}] \leavevmode
int/None: The block size:
\begin{itemize}
\item {} 
None - Failed!

\item {} 
int  - The block size.

\end{itemize}

\end{description}

Usage:

\begin{sphinxVerbatim}[commandchars=\\\{\}]
\PYG{g+gp}{\PYGZgt{}\PYGZgt{}\PYGZgt{} }\PYG{n}{compute\PYGZus{}block\PYGZus{}size}\PYG{p}{(}\PYG{o}{\PYGZlt{}}\PYG{n}{stdoutFromDiskutil}\PYG{o}{\PYGZgt{}}\PYG{p}{)}
\end{sphinxVerbatim}

\end{fulllineitems}

\index{get\_block\_size() (in module getdevinfo.macos)@\spxentry{get\_block\_size()}\spxextra{in module getdevinfo.macos}}

\begin{fulllineitems}
\phantomsection\label{\detokenize{macos:getdevinfo.macos.get_block_size}}\pysiglinewithargsret{\sphinxcode{\sphinxupquote{getdevinfo.macos.}}\sphinxbfcode{\sphinxupquote{get\_block\_size}}}{\emph{disk}}{}
\sphinxstylestrong{Public}

This function uses the diskutil info command to get the block size
of the given device.
\begin{description}
\item[{Args:}] \leavevmode\begin{description}
\item[{disk (str):     The partition/device that}] \leavevmode
we want the block size for.

\end{description}

\item[{Returns:}] \leavevmode
int/None. The block size.
\begin{itemize}
\item {} 
None - Failed!

\item {} 
int  - The block size.

\end{itemize}

\end{description}

Usage:

\begin{sphinxVerbatim}[commandchars=\\\{\}]
\PYG{g+gp}{\PYGZgt{}\PYGZgt{}\PYGZgt{} }\PYG{n}{block\PYGZus{}size} \PYG{o}{=} \PYG{n}{get\PYGZus{}block\PYGZus{}size}\PYG{p}{(}\PYG{o}{\PYGZlt{}}\PYG{n}{aDeviceName}\PYG{o}{\PYGZgt{}}\PYG{p}{)}
\end{sphinxVerbatim}

\end{fulllineitems}

\index{get\_boot\_record() (in module getdevinfo.macos)@\spxentry{get\_boot\_record()}\spxextra{in module getdevinfo.macos}}

\begin{fulllineitems}
\phantomsection\label{\detokenize{macos:getdevinfo.macos.get_boot_record}}\pysiglinewithargsret{\sphinxcode{\sphinxupquote{getdevinfo.macos.}}\sphinxbfcode{\sphinxupquote{get\_boot\_record}}}{\emph{disk}}{}
Not yet implemented, returns (“Unknown”, “Unknown”).

\end{fulllineitems}

\index{get\_capabilities() (in module getdevinfo.macos)@\spxentry{get\_capabilities()}\spxextra{in module getdevinfo.macos}}

\begin{fulllineitems}
\phantomsection\label{\detokenize{macos:getdevinfo.macos.get_capabilities}}\pysiglinewithargsret{\sphinxcode{\sphinxupquote{getdevinfo.macos.}}\sphinxbfcode{\sphinxupquote{get\_capabilities}}}{\emph{disk}}{}
Not yet implemented, returns “Unknown”

\end{fulllineitems}

\index{get\_capacity() (in module getdevinfo.macos)@\spxentry{get\_capacity()}\spxextra{in module getdevinfo.macos}}

\begin{fulllineitems}
\phantomsection\label{\detokenize{macos:getdevinfo.macos.get_capacity}}\pysiglinewithargsret{\sphinxcode{\sphinxupquote{getdevinfo.macos.}}\sphinxbfcode{\sphinxupquote{get\_capacity}}}{}{}
Private, implementation detail.

This function gets the capacity of the disk currently referenced in
the diskutil info output we’re storing. You can’t really use this standalone.
Also rounds it to a human-readable form, and returns both sizes.
\begin{description}
\item[{Returns:}] \leavevmode
tuple (string, string). The sizes (bytes, human-readable):
\begin{itemize}
\item {} 
(“Unknown”, “Unknown”)     - Couldn’t find them.

\item {} 
Anything else              - The sizes.

\end{itemize}

\end{description}

Usage:

\begin{sphinxVerbatim}[commandchars=\\\{\}]
\PYG{g+gp}{\PYGZgt{}\PYGZgt{}\PYGZgt{} }\PYG{n}{raw\PYGZus{}size}\PYG{p}{,} \PYG{n}{human\PYGZus{}size} \PYG{o}{=} \PYG{n}{get\PYGZus{}capacity}\PYG{p}{(}\PYG{p}{)}
\end{sphinxVerbatim}

\end{fulllineitems}

\index{get\_description() (in module getdevinfo.macos)@\spxentry{get\_description()}\spxextra{in module getdevinfo.macos}}

\begin{fulllineitems}
\phantomsection\label{\detokenize{macos:getdevinfo.macos.get_description}}\pysiglinewithargsret{\sphinxcode{\sphinxupquote{getdevinfo.macos.}}\sphinxbfcode{\sphinxupquote{get\_description}}}{\emph{disk}}{}
Private, implementation detail.

This function generates a human-readable description of the given disk.
\begin{description}
\item[{Args:}] \leavevmode
disk (str):   Name of a device/partition.

\item[{Returns:}] \leavevmode\begin{description}
\item[{string. The description: This may contain various bits of info, or not,}] \leavevmode
depending on what macOS knows about the disk.

\end{description}

\end{description}

Usage:

\begin{sphinxVerbatim}[commandchars=\\\{\}]
\PYG{g+gp}{\PYGZgt{}\PYGZgt{}\PYGZgt{} }\PYG{n}{description} \PYG{o}{=} \PYG{n}{get\PYGZus{}description}\PYG{p}{(}\PYG{o}{\PYGZlt{}}\PYG{n}{aDisk}\PYG{o}{\PYGZgt{}}\PYG{p}{)}
\end{sphinxVerbatim}

\end{fulllineitems}

\index{get\_device\_info() (in module getdevinfo.macos)@\spxentry{get\_device\_info()}\spxextra{in module getdevinfo.macos}}

\begin{fulllineitems}
\phantomsection\label{\detokenize{macos:getdevinfo.macos.get_device_info}}\pysiglinewithargsret{\sphinxcode{\sphinxupquote{getdevinfo.macos.}}\sphinxbfcode{\sphinxupquote{get\_device\_info}}}{\emph{disk}}{}
Private, implementation detail.

This function gathers and assembles information for devices (whole disks).
It employs some simple logic and the other functions defined in this
module to do its work.
\begin{description}
\item[{Args:}] \leavevmode
disk (str): The name of a device, without the leading /dev. eg: disk1

\item[{Returns:}] \leavevmode
string.     The name of the device.

\end{description}

Usage:

\begin{sphinxVerbatim}[commandchars=\\\{\}]
\PYG{g+gp}{\PYGZgt{}\PYGZgt{}\PYGZgt{} }\PYG{n}{host\PYGZus{}disk} \PYG{o}{=} \PYG{n}{get\PYGZus{}device\PYGZus{}info}\PYG{p}{(}\PYG{o}{\PYGZlt{}}\PYG{n}{aNode}\PYG{o}{\PYGZgt{}}\PYG{p}{)}
\end{sphinxVerbatim}

\end{fulllineitems}

\index{get\_file\_system() (in module getdevinfo.macos)@\spxentry{get\_file\_system()}\spxextra{in module getdevinfo.macos}}

\begin{fulllineitems}
\phantomsection\label{\detokenize{macos:getdevinfo.macos.get_file_system}}\pysiglinewithargsret{\sphinxcode{\sphinxupquote{getdevinfo.macos.}}\sphinxbfcode{\sphinxupquote{get\_file\_system}}}{\emph{disk}}{}
Not yet implemented, returns “Unknown”.

\end{fulllineitems}

\index{get\_id() (in module getdevinfo.macos)@\spxentry{get\_id()}\spxextra{in module getdevinfo.macos}}

\begin{fulllineitems}
\phantomsection\label{\detokenize{macos:getdevinfo.macos.get_id}}\pysiglinewithargsret{\sphinxcode{\sphinxupquote{getdevinfo.macos.}}\sphinxbfcode{\sphinxupquote{get\_id}}}{\emph{disk}}{}
Not yet implemented, returns “Unknown”.

\end{fulllineitems}

\index{get\_info() (in module getdevinfo.macos)@\spxentry{get\_info()}\spxextra{in module getdevinfo.macos}}

\begin{fulllineitems}
\phantomsection\label{\detokenize{macos:getdevinfo.macos.get_info}}\pysiglinewithargsret{\sphinxcode{\sphinxupquote{getdevinfo.macos.}}\sphinxbfcode{\sphinxupquote{get\_info}}}{}{}
This function is the macOS-specific way of getting disk information.
It makes use of the diskutil list, and diskutil info commands to gather
information.

It uses the other functions in this module to acheive its work, and
it \sphinxstylestrong{doesn’t} return the disk infomation. Instead, it is left as a
global attribute in this module (DISKINFO).
\begin{description}
\item[{Raises:}] \leavevmode
Nothing, hopefully, but errors have a small chance of propagation
up to here here. Wrap it in a try:, except: block if you are worried.

\end{description}

Usage:

\begin{sphinxVerbatim}[commandchars=\\\{\}]
\PYG{g+gp}{\PYGZgt{}\PYGZgt{}\PYGZgt{} }\PYG{n}{get\PYGZus{}info}\PYG{p}{(}\PYG{p}{)}
\end{sphinxVerbatim}

\end{fulllineitems}

\index{get\_partition\_info() (in module getdevinfo.macos)@\spxentry{get\_partition\_info()}\spxextra{in module getdevinfo.macos}}

\begin{fulllineitems}
\phantomsection\label{\detokenize{macos:getdevinfo.macos.get_partition_info}}\pysiglinewithargsret{\sphinxcode{\sphinxupquote{getdevinfo.macos.}}\sphinxbfcode{\sphinxupquote{get\_partition\_info}}}{\emph{disk}, \emph{host\_disk}}{}
Private, implementation detail.

This function gathers and assembles information for partitions.
It employs some simple logic and the other functions defined in this
module to do its work.
\begin{description}
\item[{Args:}] \leavevmode\begin{description}
\item[{disk (str):         The name of a partition, without the leading}] \leavevmode
/dev. eg: disk1s1

\item[{host\_disk (str):    The “parent” or “host” device. eg: for}] \leavevmode
/dev/disk1s1, the host disk would be /dev/disk1.
Used to organise everything nicely in the
disk info dictionary.

\end{description}

\item[{Returns:}] \leavevmode
string.     The name of the partition.

\end{description}

Usage:

\begin{sphinxVerbatim}[commandchars=\\\{\}]
\PYG{g+gp}{\PYGZgt{}\PYGZgt{}\PYGZgt{} }\PYG{n}{volume} \PYG{o}{=} \PYG{n}{get\PYGZus{}device\PYGZus{}info}\PYG{p}{(}\PYG{o}{\PYGZlt{}}\PYG{n}{aDisk}\PYG{o}{\PYGZgt{}}\PYG{p}{,} \PYG{o}{\PYGZlt{}}\PYG{n}{aHostDisk}\PYG{o}{\PYGZgt{}}\PYG{p}{)}
\end{sphinxVerbatim}

\end{fulllineitems}

\index{get\_partitioning() (in module getdevinfo.macos)@\spxentry{get\_partitioning()}\spxextra{in module getdevinfo.macos}}

\begin{fulllineitems}
\phantomsection\label{\detokenize{macos:getdevinfo.macos.get_partitioning}}\pysiglinewithargsret{\sphinxcode{\sphinxupquote{getdevinfo.macos.}}\sphinxbfcode{\sphinxupquote{get\_partitioning}}}{\emph{disk}}{}
Not yet implemented, returns “Unknown”.

\end{fulllineitems}

\index{get\_product() (in module getdevinfo.macos)@\spxentry{get\_product()}\spxextra{in module getdevinfo.macos}}

\begin{fulllineitems}
\phantomsection\label{\detokenize{macos:getdevinfo.macos.get_product}}\pysiglinewithargsret{\sphinxcode{\sphinxupquote{getdevinfo.macos.}}\sphinxbfcode{\sphinxupquote{get\_product}}}{\emph{disk}}{}
Private, implementation detail.

This function gets the product of the given disk.
\begin{description}
\item[{Args:}] \leavevmode
disk (str):   Name of a device/partition.

\item[{Returns:}] \leavevmode
string. The product:
\begin{itemize}
\item {} 
“Unknown”     - Couldn’t find it.

\item {} 
Anything else - The product.

\end{itemize}

\end{description}

Usage:

\begin{sphinxVerbatim}[commandchars=\\\{\}]
\PYG{g+gp}{\PYGZgt{}\PYGZgt{}\PYGZgt{} }\PYG{n}{product} \PYG{o}{=} \PYG{n}{get\PYGZus{}product}\PYG{p}{(}\PYG{o}{\PYGZlt{}}\PYG{n}{aDisk}\PYG{o}{\PYGZgt{}}\PYG{p}{)}
\end{sphinxVerbatim}

\end{fulllineitems}

\index{get\_uuid() (in module getdevinfo.macos)@\spxentry{get\_uuid()}\spxextra{in module getdevinfo.macos}}

\begin{fulllineitems}
\phantomsection\label{\detokenize{macos:getdevinfo.macos.get_uuid}}\pysiglinewithargsret{\sphinxcode{\sphinxupquote{getdevinfo.macos.}}\sphinxbfcode{\sphinxupquote{get\_uuid}}}{\emph{disk}}{}
Not yet implemented, returns “Unknown”.

\end{fulllineitems}

\index{get\_vendor() (in module getdevinfo.macos)@\spxentry{get\_vendor()}\spxextra{in module getdevinfo.macos}}

\begin{fulllineitems}
\phantomsection\label{\detokenize{macos:getdevinfo.macos.get_vendor}}\pysiglinewithargsret{\sphinxcode{\sphinxupquote{getdevinfo.macos.}}\sphinxbfcode{\sphinxupquote{get\_vendor}}}{\emph{disk}}{}
Private, implementation detail.

This function gets the vendor of the given disk.
\begin{description}
\item[{Args:}] \leavevmode
disk (str):   Name of a device/partition.

\item[{Returns:}] \leavevmode
string. The vendor:
\begin{itemize}
\item {} 
“Unknown”     - Couldn’t find it.

\item {} 
Anything else - The vendor.

\end{itemize}

\end{description}

Usage:

\begin{sphinxVerbatim}[commandchars=\\\{\}]
\PYG{g+gp}{\PYGZgt{}\PYGZgt{}\PYGZgt{} }\PYG{n}{vendor} \PYG{o}{=} \PYG{n}{get\PYGZus{}vendor}\PYG{p}{(}\PYG{o}{\PYGZlt{}}\PYG{n}{aDisk}\PYG{o}{\PYGZgt{}}\PYG{p}{)}
\end{sphinxVerbatim}

\end{fulllineitems}

\index{is\_partition() (in module getdevinfo.macos)@\spxentry{is\_partition()}\spxextra{in module getdevinfo.macos}}

\begin{fulllineitems}
\phantomsection\label{\detokenize{macos:getdevinfo.macos.is_partition}}\pysiglinewithargsret{\sphinxcode{\sphinxupquote{getdevinfo.macos.}}\sphinxbfcode{\sphinxupquote{is\_partition}}}{\emph{disk}}{}
Private, implementation detail.

This function determines if a disk is a partition or not.
\begin{description}
\item[{Args:}] \leavevmode
disk (str):   Name of a device/partition.

\item[{Returns:}] \leavevmode
bool:
\begin{itemize}
\item {} 
True  - Is a partition.

\item {} 
False - Not a partition.

\end{itemize}

\end{description}

Usage:

\begin{sphinxVerbatim}[commandchars=\\\{\}]
\PYG{g+gp}{\PYGZgt{}\PYGZgt{}\PYGZgt{} }\PYG{n}{is\PYGZus{}a\PYGZus{}partition} \PYG{o}{=} \PYG{n}{is\PYGZus{}partition}\PYG{p}{(}\PYG{o}{\PYGZlt{}}\PYG{n}{aDisk}\PYG{o}{\PYGZgt{}}\PYG{p}{)}
\end{sphinxVerbatim}

\end{fulllineitems}



\chapter{Indices and tables}
\label{\detokenize{index:indices-and-tables}}\begin{itemize}
\item {} 
\DUrole{xref,std,std-ref}{genindex}

\item {} 
\DUrole{xref,std,std-ref}{modindex}

\item {} 
\DUrole{xref,std,std-ref}{search}

\end{itemize}


\renewcommand{\indexname}{Python Module Index}
\begin{sphinxtheindex}
\let\bigletter\sphinxstyleindexlettergroup
\bigletter{g}
\item\relax\sphinxstyleindexentry{getdevinfo.cygwin}\sphinxstyleindexpageref{cygwin:\detokenize{module-getdevinfo.cygwin}}
\item\relax\sphinxstyleindexentry{getdevinfo.getdevinfo}\sphinxstyleindexpageref{getdevinfo:\detokenize{module-getdevinfo.getdevinfo}}
\item\relax\sphinxstyleindexentry{getdevinfo.linux}\sphinxstyleindexpageref{linux:\detokenize{module-getdevinfo.linux}}
\item\relax\sphinxstyleindexentry{getdevinfo.macos}\sphinxstyleindexpageref{macos:\detokenize{module-getdevinfo.macos}}
\end{sphinxtheindex}

\renewcommand{\indexname}{Index}
\printindex
\end{document}